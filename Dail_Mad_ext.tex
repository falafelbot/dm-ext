\documentclass{article}

\usepackage[a4paper, hmargin={2.4cm,2.4cm}, vmargin={2.4cm,2.4cm}]{geometry}
\usepackage[mathlines,displaymath]{lineno}
\usepackage{setspace}
\usepackage{amsmath}
\usepackage{graphicx}
\usepackage[round,colon,authoryear]{natbib}
\usepackage{color, soul}
\usepackage{fancyvrb}
\usepackage{parskip}

\bibliographystyle{ecology}

\begin{document}

\abstract{
Models of population dynamics play a central role in theoretical and
applied ecology where they are used for purposes such as testing
hypotheses about density dependence and predicting species' responses
to future environmental change or management actions. State-space
models, are widely used for these purposes, and they allow for
estimating parameters of classical growth models, such as exponential,
logistic, and Gompertz, while accounting for demographic stochasticity
and observation error. Conventional state-space models, however, have
three important limitations: (1) the parameters are not identifiable
in many common situations, (2) they do not admit spatial variation in
population dynamics, and (3) there is no clear interpretation of the
observation error. We demonstrate how each of these problems can be
resolved using a class of hierarchical models for spatially-replicated
time-series data recently proposed by \citet{dail_madsen:2011}. We
expand this class of models to accommodate classical growth
models, zero-inflation, and random effects such as observer-specific
detection probabilities. Furthermore, we describe classical and
Bayesian methods of parameter estimation. These new developments will
allow researchers to apply these methods to address important
questions regarding the factors affecting spatial and temporal
variation in abundance. We demonstrate these developments by analyzing
data from the North American Breeding Bird Survey. Code to fit these
models using the R package unmarked and JAGS is also provided.
}

\textbf{Key words}: abundance, Dail and Madsen model, density-dependence,
Gompertz model, immigration, open population point count models,
random observer effects, range, Ricker model, zero-inflated

\newpage

Theoretical ecology requires robust models for testing hypotheses
regarding density dependence and spatial variation in population
dynamics. Such models are also required in applied ecological research
for predicting species' responses to environmental change or
management actions. For example, many researchers are currently
investigating the influence of weather on population dynamics so that
predictions can be made about the effects of future climate
change. Several complicating factors must be confronted when
addressing these questions. First, deterministic models of population
dynamics are virtually always inadequate due to ``process variation'',
the inherent stochasticity in demographic parameters and environmental
conditions. Second, abundance---the natural state variable in studies
of population dynamics---can rarely be observed perfectly in field
studies because of ``observation error'', such as imperfect
detection. Failure to account for  and heterogeneity in observation
error can bias the estimators of abundance and related parameters.

State-space models have become the most widespread approach for
studying population dynamics while accounting for process variation
and observation error (de Valpine and Hastings 2002, Buckland et
al. 2004, Dennis et al. 2006). Conventional state-space models are
simply time-series models in which the true state of the system is
observed imperfectly. Originally developed for xxx (cite Kalman filter
papers), the ability to model both the ecological state process and
the observation process has made them relevant in numerous ecological
applications [need to clean this up]. State-space models have been
defined differently by different authors, but we follow Buckland et
al. (2004) using the definition : ``xxx'' . Typically, the ecological
process is described by a Markovian model that includes a
deterministic description of population change as well as two random
sources of variation for ``process variation''. The model for the
observation process is typically phenomenological in the sense that
observation error is not defined explicitly. As an example, let  be
the abundance of a species during year  , and   be the observed data,
which differs from  due to observation error ( ). A simple state space
model that includes exponential growth and process variation ( ) could
be written as
	 	(1)
Here   is the intrinsic rate of increase. Typically, both random
effects are assumed to follow normal distributions:   and   (de
Valpine and Hastings 2002, Dennis et al. 2006). Even though this is
the most widespread approach for modeling population dynamics using
time-series data, several problems are immediately evident. First, if
abundance is a non-negative integer, as it always is, there is clearly
a problem with this formulation because the normally distributed
random effects violate this constraint. To avoid this issue,
ecologists often replace   with   where A is the area surveyed---i.e.,
abundance is replaced by the natural log of population density, which
is assumed to normally distributed. This is desirable because this
allows for the fitting of models assuming normally distributed
residuals, which are computationally much less intensive than most
alternatives. However, this transformation is problematic since
abundance may be zero and thus . Generally, zeros are replaced with
some small, but positive number, although the effect of this is rarely
discussed.

The problems mentioned thus far are minor in comparison to the more
serious issue that the parameters of the model are often not
identifiable. Specifically, the Markovian nature of the model implies
the following likelihood
\[
LIKELIHOOD
\]
The first term is the probability density for the first observation in
the time series,  , yet it is intuitively not possible to estimate the
parameter(s) of this distribution using only a single
observation. This problem has resulted in several papers with
interesting names like ``multi-modal likelihoods and ridges etc...''
The workaround here is to assume that the population is at equilibrium
so that that the x can be replaced with the equilibrium
distribution. However, assuming equilibrium defeats the objective of
many studies of population dynamics, namely determining if and why a
population is at equilibrium.

A third problem with these models is that they do not admit
spatial variation. This is not a fair criticism because this
lies beyond the scope of traditional state-space models, but
it does restrict the utility of their since it is typically
impossible assess the impacts of factors such as habitat
fragmentation or climate change on a population without
considering spatial variation. Indeed many populations are
regulated by spatial processes such as source-sink dynamics.

Dail and Madsen (2011) developed a model (henceforth the DM model)
that resolves each of these problems with traditional state-space
models. The DM model allows for inference about spatial and temporal
dynamics in abundance while accounting for observation error using
only spatially and temporally replicated count data.  Their model is
an extension of the closed-population N-mixture model (Royle
2004)(2004), which was designed specifically to address the problem of
modeling spatial variation in abundance when detection probability is
less than unity. The DM extension relaxes the assumption of population
closure and includes explicit parameters describing population change
as a first-order Markovian process. [more details]

[applications / potential etc... ] The DM model is relevant to a huge
number of problems confronting basic and applied ecological
research. Answering questions about climate change etc... require
spatially and temporally extensive datasets such as result from the
North American Breeding Bird Survey (BBS) (Robbins et
al. 1986). However, most of the datasets of the appropriate scale are
collected by volunteers and using protocols that are not amenable to
traditional approaches of modeling abundance and detection probability
based on standard capture-recapture methods. Instead they have, at
best, simple count data for which, when it can be interpreted as
counts of unique individuals, the DM model can be applied.. Yadi ya

Although the DM model was designed explicitly for these purposes, the
original formulation of the model makes strict distributional
assumptions and does not acknowledge many of the sources of variation
inherent to existing ecological time-series data.  We propose
variations on these models that are likely to be more realistic and
useful in many cases.  We demonstrate the usefulness of these new
models with simulated and real BBS data.









\bibliography{Dail_Mad_ext}

\end{document}
