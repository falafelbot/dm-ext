\documentclass[12pt,english]{article}
\usepackage[letterpaper,hmargin={2.4cm,2.4cm},vmargin={2.4cm,2.4cm}]{geometry}
\usepackage{setspace}
\usepackage{amsfonts}
\usepackage{bm}
\usepackage[round,comma,authoryear]{natbib}
\usepackage{color, soul}
\usepackage{fancyvrb}
\usepackage{verbatim}
\usepackage[format=plain,justification=raggedright,labelsep=period,font={stretch=1.9}]{caption}
\usepackage[pdftex,hidelinks]{hyperref} 
%\usepackage[compact]{titlesec}
%\titleformat{\section}[display]{\normalfont\bfseries}{}{0em}{\centering}
%\titleformat{\subsection}[display]{\normalfont\itshape}{}{0em}{\centering}
\bibliographystyle{ecology}
%\setlength{\bibsep}{0.0pt}
\bibpunct{(}{)}{,}{a}{}{,}
%\setlength{\jot}{-1.2ex}
\usepackage{etoolbox}
\AtBeginEnvironment{tabular}{\doublespacing}
\makeatletter
\let\@fnsymbol\@arabic
\makeatother
\title{Improved state-space models for inference about 
spatial and temporal variation in abundance from count data: Response to comments}

\author{Jeffrey A. Hostetler\thanks{Migratory Bird Center, Smithsonian 
Conservation Biology Institute, National Zoological Park, MRC 5503, 
Washington, DC 20013-7012} \textsuperscript{,}\footnote{Email: \href{mailto:hostetlerj@si.edu}{hostetlerj@si.edu}}
   \and Richard B. Chandler\footnote{USGS Patuxent Wildlife Research Center 
        Laurel, MD 20708} \textsuperscript{,}\thanks{Current address: Warnell School of
     Forestry and Natural Resources, University of Georgia}        
}
\date{} %remove date
\begin{document}
\maketitle
\vspace{-1cm}
\begin{spacing}{1.9}
\begin{flushleft}
\renewcommand*\thetable{R\arabic{table}}
\renewcommand*\thefigure{R\arabic{figure}}
\renewcommand*\theequation{R\arabic{equation}}

\section*{General Comments}
\label{sec:general}

In this submission, Hostetler and Chandler extend the so-called Dall and Madsen (DM) 
model for temporally and spatially-replicated animal point counts to include additional 
forms of process dynamics (e.g. exponential growth, density dependence) and to better 
cope with common features of ecological datasets (e.g. zero inflation, environmental noise, 
variation in observation error). The paper is well written, should be understandable by ecologists, 
and is reasonably motivated, the authors having situated their modeling approach within the 
context of state-space models for ecological time series (see Specific Comments for some 
musings on this topic). The authors also include requisite computing code to reconstruct the 
analysis within the R package unmarked, which is a nice touch and should make implementation 
of some of these methods easier for ecologists.

%\hfill\begin{minipage}{\dimexpr\textwidth-25pt}
\textit{Thank you for your positive comments.}
%\end{minipage}

Despite these positives, I do have several reservations about the manuscript. First, the notion that 
one can estimate so many quantities with simple count data seems to me to be something like trying 
to squeeze blood out of a turnip. If distributional assumptions hold, the authors have shown that a 
good deal of parameters (initial state distribution, population growth rates, immigration) can be 
estimated unbiasedly and have other good properties (coverage etc.). However, what if those 
distributional assumptions are violated?

In an ideal world, it seems to me that simulation studies should include two components: a 
'proof of concept' type study to demonstrate parameter identifiability and to make sure the code 
isn't buggy (using the exact distributional forms to simulate data as are used in estimation), and 
(2) a robustness study to examine performance of the proposed model under suspected assumption 
violations. I think the ecological estimation literature is too often characterized by studies that 
employ (1) and not (2) (I am guilty of this myself). However, the need for a 'robustness' study seems 
particularly germane in this case, given the large number of assumptions to proceed with inference here. 
I suspect the authors must have similar feelings, given some of the statements in the discussion about 
the importance of including auxiliary data on detection probability, vital rates, etc.

%\vspace{1cm}\hfill\begin{minipage}{\dimexpr\textwidth-25pt}
\textit{We have added a robustness simulation section. We included three additional simulation models:
DM with geometric-recruitment dynamics (underdispersed compared to Poisson), a Leslie matrix (underdispersed),
and a Gompertz-logistic with two generations per year (overdispersed). We fit the first two sets of simulated
data with the DM geometric-recruitment and our exponential version in unmarked, and the third with
a Gompertz-logistic model with one generation per year (unmarked and JAGS) and two generations
per year (JAGS).  In addition to comparing model estimates, we also examined how well AIC did at
selecting the more appropriate model in the first two cases.}
%\end{minipage}\vspace{1cm}

I am also a bit troubled by the use of a Poisson process to encompass process error. This assumption 
in itself is a very strong one (see Specific Comments below), which seems to me to be an additional 
argument for conducting a robustness study. What happens for instance if you use a Leslie-matrix 
type model on individual sites and then try to fit the Poisson process model? Do estimates come out 
near where they should? If not, is including an informative prior on detection probability (as if one 
had auxiliary data) sufficient to achieve good estimator properties?

%\vspace{1cm}\hfill\begin{minipage}{\dimexpr\textwidth-25pt}
\textit{In the Leslie matrix and geometric-recruitment simulations, estimates for $\Lambda$ and $p$ from
the exponential model were biased, and in the Gompertz-logistic simulation, all estimates from the single
generation models were biased.  However, between the models presented in DM and here (including new 
overdispersion models), we cover almost
all cases of density-dependence and dispersion (see table).  Furthermore, AIC does a good job of distinguishing
between Poisson distributed and underdispersed data.}
%\end{minipage}\vspace{1cm}

\begin{table}[t]
  \centering
\caption{Population dynamics options covered by improved state-space models.}  
\begin{tabular}{lccc}
    \hline
Dispersion & Density-Independent & Density-Dependent  \\    
\hline
    Underdispersed	        &DM	&Not covered yet  \\
    Poisson-distributed	        &Here	&Here   \\
    Overdispersed		 &Here	&Here    \\
    \hline
  \end{tabular}
\end{table}


\section*{Specific Comments}

1. Just pontificating here: I wonder if the use of the term “state-space" is becoming a little over-extended 
in the ecological literature. For instance, the generic state space model written in Eq. 1 certainly is a 
state space model in the sense it has Gaussian error and can be fitted using the Kalman filter. One could 
argue that the de Valpine and Hastings (2002) model is analagous to a state space model, and they seem 
to have borrowed the terminology when naming their model. The models in the present paper share 
several features with state-space models: they are Markovian time series models with both process and 
observation error. On the other hand, I'm not sure it makes sense to talk of the Kery et al. (2009) model 
as a state space model at all. It doesn't have Gaussian error, can't be fit using a forward-backward type 
algorithm, and doesn't even have serial dependence. When does a “state space” model just become a 
hierarchical time series model??

%%[Richard, what do you think?  Should we say that Kery 2009 might not be state-space to appease this person?]
\textit{We agree that some authors use the term ``state-space'' so broadly that its usefulness may be diminshed. However, we believe that the definition of state-space models that we provide in our introduction is sufficiently narrow to exclude many classes of hierarcical models. Having said that, we agree that the K\'{e}ry et al. model doesn't fit within our definition because it isn't Markovian. This is now mentioned in the introduction.}

2. Is there a justification for using the Poisson process error model in Eqn 5 and throughout the remainder 
of the paper? This does not seem like an innocuous distributional choice, and it seems like real data could 
easily be over- or under-dispersed relative to the Poisson (e.g. with long-lived mammals or insects, respectively). 
Error associated with the binomial and Poisson distribution are drastically different unless the binomial success 
probability is really small and the index is large. I'm just not seeing any real justification other than modeling convenience.

%\vspace{1cm}\hfill\begin{minipage}{\dimexpr\textwidth-25pt}
\textit{When we have compared our models to the DM models with real data, the estimates from our models
are more plausible and the AIC lower.  That's our main justification - the data support the Poisson process error.
In addition, we now present models that account for overdispersion, and the original DM models account for
some degree of underdispersion (see table).}
%\end{minipage}\vspace{1cm}

3. Simulation study. There needs to be more detail here. Was process error included in simulations? If so, was it 
induced with the same Poisson assumption as in estimation? I guess what I'm getting at here is that I'm dubious 
on whether the Poisson distribution is appropriate, so would be interested in seeing results where a different process 
error form is used in simulation that estimation. For instance, what does estimator performance look like when data 
are simulated with a binomial survival model and a Poisson immigration model as in the original DM model

%\vspace{1cm}\hfill\begin{minipage}{\dimexpr\textwidth-25pt}
\textit{Our ``proof of concept'' simulations did use Poisson process error.  We have added one of the original DM
models to our robustness simulation section.}
%\end{minipage}\vspace{1cm}

\section*{Minor Comments/Typos}
1. Equation 1. As written, lines 1 and 3 imply that ζ1 = 0 (no sampling error in year 1). Given that there are 
other alternatives (e.g. the authors allude to using a stationarity assumption in year 1), would it make more 
sense to just specify an initial condition like N1 = x0 and state what the choices of x0 can be later (e.g. lines 95-97)?

[Richard, what do you think?   Jeff, yes I think we should make this change. ]

2. Lines 83-84. Instead of speaking of elusiveness, it may be better to distinguish between availability 
and detection probability given available as you do in the discussion- I think this way of looking at things is a little more general.

\textit{Thank you for the suggestion!  We have amended the text (lines X).}

3. Line 91. Ignoring spatial variation (with respect to habitat covariates as well as spatial autocorrelation) has great potential for underestimating variances

\textit{This is a good point.  We have amended the text (lines X).} [OOPS, I THINK I TOOK THIS OUT. WILL YOU ADD IT BACK IN?]

4. Line 132. Denoted what?

\textit{Denoted Ni,1.  We have fixed this (line X).}

5. Line 150. I think these types of models would benefit from adding spatial autocorrelation via random effects at this stage

\textit{We agree that this would be a useful extension of these models, but believe it is beyond the scope of this manuscript.}

6. Line 192 'to in'

\textit{We have corrected this (line X).}

7. Line 240. `Three variations suggest themselves' - sounds like spontaneous generation!

\textit{We have taken ownership of these ideas (lines X).}

8. Line 294 simulations not simulation

\textit{We have corrected this (line X).}
\end{flushleft}
\end{spacing}
\end{document}
