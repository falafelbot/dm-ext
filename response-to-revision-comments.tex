\documentclass[12pt,english]{article}
\usepackage[letterpaper,hmargin={2.4cm,2.4cm},vmargin={2.4cm,2.4cm}]{geometry}
\usepackage{setspace}
\usepackage{amsfonts}
\usepackage{bm}
\usepackage[round,comma,authoryear]{natbib}
\usepackage{color, soul}
\usepackage{fancyvrb}
\usepackage{verbatim}
\usepackage[format=plain,justification=raggedright,labelsep=period,font={stretch=1.9}]{caption}
\usepackage[pdftex,hidelinks]{hyperref} 
\bibliographystyle{ecology}
\bibpunct{(}{)}{,}{a}{}{,}
\usepackage{etoolbox}
\AtBeginEnvironment{tabular}{\doublespacing}
\makeatletter
\let\@fnsymbol\@arabic
\makeatother
\title{Improved state-space models for inference about 
spatial and temporal variation in abundance from count data: Response to revision comments}

\author{Jeffrey A. Hostetler
   \and Richard B. Chandler    
}
\date{} 
\begin{document}
\maketitle
\vspace{-1cm}
\begin{spacing}{1.9}
\begin{flushleft}
\renewcommand*\thetable{R\arabic{table}}
\renewcommand*\thefigure{R\arabic{figure}}
\renewcommand*\theequation{R\arabic{equation}}

\section*{Reviewer 1 (Comments to Author):} 
\label{sec:rev1}

\subsection*{General Comments}
In this paper, the authors propose to extend the \citet{dail_madsen:2011} model to 
include classical models of population growth. They also propose extensions to 
deal with zero-inflated data, immigration and variation in observation errors.

This work is novel and important because it proposes to estimate population 
growth characteristics by accounting for imperfect detection of individuals. Otherwise, 
actually, population growth models are embedded in state-based models that assume 
normally distributed errors, as a consequence probability of detection cannot be estimated. 
Besides, in some models of population growth, the errors of observations are not taken 
into account.

The authors check for the reliability of their model to recover their parameters by using a 
simulation study. The simulation study covers a large range of different population 
dynamics as well as additional cases where the simulated data deviates from the assumption 
of the model used for the analysis. In addition, to the simulation study, the models are 
applied to real data.

\vspace{0.5cm}
\textit{Thank you for your positive comments!}
\vspace{0.5cm}

In a general way, effort should be put on tightening the text and better focus
on the main goal of the paper, that is about including models of population
growth in a framework that account for imperfect detection of individuals. The
choice of the model of population growth might be subjective and/or motivated by
the knowledge of the life history of the species under study. Even thought, some
studies, proposed some procedures based on model selection to discriminate among
the different models of population growth. Hence, the Ms would benefit of a
better description of the underlying ecological assumptions of the different
models of population growth. For example, the exponential model describes
population growth in an unlimited environment, where $r$ is the maximum per capita
of increase, etc… . In Ricker model, the density-dependence becomes stronger at
higher densities, due to the exponential function relating to carrying capacity.
The Gompertz model is similar to the Ricker model except that there is a
constant linear decrease in the instantaneous growth rate ($r$) as the natural
logarithm of population increase, etc…..

\vspace{0.5cm}
\textit{[We have described the classical population growth models in more 
detail (lines ).]}
\vspace{0.5cm}

\subsection*{Specific Comments}
I am unclear about the title of the manuscript because it does not reflect the real aim of the
paper, which is well stated by the authors, lines 417-418: ``one of the primary aims of our
paper was to demonstrate how classical population growth models can be embedded in the
DM model''.

I find that basing the introduction upon the notion of state-based model is rather confusing
regarding the real goal of the paper (see above). The Dail and Madsen model and state-space
models are defined in two different ``statistical worlds''. On the one hand, the constant
Dail and Madsen model (2011) has a Markov property so the variables can only take on a
finite set of discrete values for a finite period of observation. The distinguish feature of
Markov chains is that the conditional probability of $x_{t}$ given the entire past depends only on
the immediately preceding variable $x_{t-1}$. In addition, the errors of observation are well
identified because they are related to imperfect detection of individuals and expressed with
a discrete probability distribution. On the other hand, the state-space model framework
provides a structure for extending time-series models to handle both observation and
process errors. The term ``state-space'' model refers to models where the process model has
an autoregressive property and where the errors have a Gaussian distribution.
Autoregressive model are defined for continuous variable, and so the errors are normally
distributed. A space-state model, being given its properties, can be estimated by using a
Kalman filter. The Kalman filter is an updating procedure whereby the prior estimate of the
state is corrected by how well the next observation can be predicted.

Even if the Dail and Madsen model also provides a structure to account for time dependence
as well as for ``process errors'' through the birth and death process and for imperfect
detection, the Dail and Madsen model and state-space models are defined in two different
``statistical worlds''. Hence, the analytical results of the Dail and Madsen model and a state-space
model would be different. That is why, in my point view, it is unclear and misleading to
relate the Dail and Madsen model to a state-space model.

\vspace{0.5cm}
Our definition of a state-space model is in agreement with the brand new
book on the subject ``Modeling Population Dynamics: Model Formulation,
Fittin, and Assessment Using State-Space Methods'' by Newman et
al (2014, Springer). They cover many state-space models that have
a first-order Markov state process and a non-Gaussian
observation process. For this reason, we think that our title connects
our work to the bigger picture, which is state-space
models of count data.  
\vspace{0.5cm}

Lines 118 - 174: the description of the model could be shortened (or moved in part in the
appendices) because it has been described now in further publications.

\vspace{0.5cm}
\textit{[Richard, any thoughts on this?  I can see the reviewer's point - there are a lot of DM
papers out now.  And this is almost 3 pages.  But I like what we've got here, and if we cut or
move it we'd have to be careful, because we refer to this section a good deal in our extensions.
How we shorten it may depend on how we respond to the DM model as state-space model critique.
Please take a stab at this if you have time and ideas on how to
shorten or move.]}
\textit{[Jeff, I don't think we should remove this, but I think we
  could put equations 2 and 4 in the text, so as to save space. We
  might also be to shorten some of the text related to covariate
  effects.]}
\vspace{0.5cm}

Line 165: \textit{``propagule rain''} is not an appropriate term here; this term is used in
Mainland/Island model to describe something like a continuous source of migrants which
can colonise an empty site. A large stable mainland population provides migrants, which is
not the case in Dail and Madsen model. Consistency with the original terminology defined in
Dail and Madsen (2011) should be kept as much as possible.

\vspace{0.5cm}
\textit{We have made the suggested change.}
\vspace{0.5cm}

Line 165: \textit{``autoregressive''} model should be kept in quote as it is p. 580 of Dail and Madsen
(2011).

\vspace{0.5cm}
\textit{We have made this change.}
\vspace{0.5cm}

Line 135: typography error in \textit{``The third process describes the relationships between''} should
be \textit{``The third process describes the relationships between.''}

\vspace{0.5cm}
\textit{We have fixed the typographic error (surplus ``the'') here.}
\vspace{0.5cm}

Line 190: More explanations on the Gompertz model should be added here, especially on
the link between the Ricker model and the Gompertz model. The original reference of the
Gompertz logistic growth equation should be added here.

\vspace{0.5cm}
\textit{[We have described the Gompertz model in more 
detail (lines ).]}

\vspace{0.5cm}

Line 199: I am unclear about the meaning of \textit{``internal and external (immigration)
contributions to population growth.''} Then, the authors proposed to account for immigration
but they do not mention anything about emigration. In the Dail and Madsen model, the
survival rate is a reduced demographic parameter that accounts for both death and
emigration. Some words, in the discussion, about how emigration is (or not) handled in this
novel extension would be welcome.

\vspace{0.5cm}
\textit{We have edited the text for clarity (lines ).  We have added discussion about emigration
(lines ).}
\vspace{0.5cm}

Line 225: I am unclear about \textit{``semelparous organisms with one generation per sampling
interval (year).''} Semelparous refers to organisms that died after reproduction, (i.e. they can
have solely one single reproduction event in their life because they die after this
reproduction event). So, semelparous organisms cannot have ``one generation per sampling
interval'' but only one generation.

Lines 225 - 229 – This paragraph needs to be clarified – semelparous organisms cannot have
more than one generation per year because they can have only one single reproduction
event in their life (see above).

\vspace{0.5cm}
\textit{We were referring to generations of the population or species, not the individual.  We have edited 
this paragraph for clarity.}
\vspace{0.5cm}

Lines 281 - 296: The paragraph about the simulation of the data needs to be clarified to
understand better how the data are simulated. Some references to the different equations
of the model should help to clarify this section. For the simulation study, the time series
which are simulated are rather long (40 years) and also the length of the time series of the
application are about 40 years. If the time series were shorter, how the estimation of the
parameters would be affected.

\vspace{0.5cm}
\textit{We have added clarification and equation references to this paragraph. We expect that
the models would not perform as well with far fewer years of data, and have commented on
this in the discussion (lines).}
\vspace{0.5cm}

Discussion: The text of the discussion needs to be reduced and tightened. The authors
should better focus on discussing the results of this study and relate them to other studies
on models of population growth.

\vspace{0.5cm}
\textit{[Richard, I commented out the two paragraphs on IPMs.  What do
  you think?                

Jeff, fine by me]}

\vspace{0.5cm}

The figure A1 and A2 describe results of the simulation based on the Ricker and Gompertz
models, and both are much more ``realistic'' than the model of exponential growth when
applied to population ecology. So, the Fig. A1 and A2 represent core results of the
manuscript and should be rather presented in the main text than in appendices.

\vspace{0.5cm}
\textit{[Between this comment and the one by the other reviewer, sounds like we should 
put all three of these figures in the main text.  If we still have to keep it to 30 pages,
that requires a page and a half of additional cutting, which is a lot.  This reviewer suggests cutting
text from DM model description and discussion.  I have worked on those (see above), but maybe we
should also consider emailing Evan Cooch about the length limit.

Jeff, I agree that we should ask Evan. If we have to modify something,
I'm in favor of moving Fig. 1 to the appendix, and inserting a table that summarizes
all the results.

]}
\vspace{0.5cm}


\section*{Reviewer 2 (Comments to Author):} 
\label{sec:rev2}
\subsection*{General Comments}
The authors build on hierarchical state-space models that describe population
dynamics proposed by Dail and Madsen (2011). The paper is clear, well-written,
and provides important advances and tests of the original DM formulation and
extensions. The authors cover a lot of ground, but to me, the principal novel
contribution of the paper is the embedding of classical population-growth and
density-dependence functions into the population dynamics model, rather than
relying on the original DM formulation that models population change as a
function of demographic (survival and recruitment) parameters. The authors also
propose a sort of hybrid set of models that combine a functional form for
population growth with a parameter representing immigration from outside of the
indexed spatial location. The paper also introduces negative binomial and
zero-inflated versions of the state model as well as models that include
demographic or environmental stochasticity. An important extension that
the authors propose that would greatly enhance the practical utility of these
models in ecological applications is the modeling of dynamics parameters as
functions of site-time specific environmental covariates or random effects. The
authors state that code will be provided for implementing models with maximum
likelihood and Bayesian McMC methods using freely available software packages,
which should facilitate the use of these models in the future. Overall, the
paper advances the state of the art of this class of models in many directions,
and the authors do a good job in highlighting both the potential and limitations
of the various model flavors.

\vspace{0.5cm}
\textit{Thank you for your positive comments!}
\vspace{0.5cm}

\subsection*{Specific comments} 
Model Extensions: Immigration models 
The immigration models are an interesting idea, combining classical growth and
density-dependence functions or the mechanistic geometric growth functions with
an additional parameter that represents immigration. As the authors note, an
important limitation of several proposed models without the immigration
parameter (including Dail and Madsen's geometric growth model) is that they do
not accommodate recovery following extinction, so the introduction of the
immigration parameter would seem a critical component in those models in cases
where local extinction is likely. I must admit, however, that I don't completely
understand how the parameters of these models are informed. For example, in the
exponential, Ricker, and Gompertz models, the r parameter includes a mixture of
local apparent survival (i.e., site fidelity + true survival) + local
recruitment (through birth processes) dynamics, while the iota parameter
supposedly reflects immigration from outside the site. How is there
information to separate recruitment components of r from immigration components
of iota? Similarly, how are both gamma and iota identifiable in the geometric
growth model? I suppose that somehow because the iota parameter does not depend
on the any of the other population parameters, it can be estimated separately.
But this is not entirely clear to me. Maybe a sentence or 2 could clarify this a
bit?

\vspace{0.5cm}
\textit{We believe the reviewer has hit upon it - the population contribution from immigration
does not depend on current abundance, whereas the contributions from $r$ in the density-dependent
models and from $\gamma$ in the geometric-recruitment model do (in different ways).  With sufficient
temporal and spatial replication, these parameters are separately estimable.  We have added text to that
effect (lines ).}
\vspace{0.5cm}

Line 347-348: As re: memory requirements, why not do multiple single-chain runs
(to yield multiple chains)?

\vspace{0.5cm}
\textit{We now have multiple chains for this model as well, and the convergence is good
with very similar results.}
\vspace{0.5cm}

Results: Why present the graphical display of simulation results for the
exponential models in the main text (Fig 1), but relegate all of the tables and
graphical summaries for other models to Appendix A? I would prefer to see all of
the summary tables (first choice) or all of the figures (but not both) in the
main text.

\vspace{0.5cm}
\textit{[Between this comment and the one by the other reviewer, sounds like we should 
put all three of these figures in the main text.]}
\vspace{0.5cm}

Fig. 1: for initial abundance on x-axis, replace little lambda with big lambda
to match text.

\vspace{0.5cm}
\textit{[Richard, can you do this, and also increase the resolution of this figure?
I don't have the simulation results from the exponential model.  Alternatively,
you could send me the results and I can replot it (I think I have your plotting code).
It looks like you used a file called process.RData to store the
results.

Jeff, let's wait to hear back from Evan. If we move this to the
appendix, I don't think these changes will be necessary.

]}
\vspace{0.5cm}

Fig. 2 legend, in parentheses could replace ``mean and'' with +/-, given that it
is already stated that the panel shows the mean.

\vspace{0.5cm}
\textit{We have made the suggested change.}
\vspace{0.5cm}

Line 358: replace 'inaccurate' with 'imprecise'.

\vspace{0.5cm}
\textit{These terms have multiple meanings and are defined differently in
different places.  We like the definition of \citet{williams_etal:2002}:
``Accuracy combines both bias and precision in an assessment of estimator
performance.''  On the same page (45) those authors describe mean squared error 
(MSE) as a measure of estimator accuracy.  Therefore root mean squared error (RMSE) and
relative RMSE are best described as measures of accuracy as well.  
In another case (line ?) we used the
word ``imprecise'' when the RMSE was high, but only after we established that
bias was low.  In this case, since bias is high, we feel it is more correct to describe
the estimates as ``inaccurate.''}
\vspace{0.5cm}

Lines 424-429: Could difficulty in estimating vital rates derives from having
survival and recruitment parameters taking similar values? For example many
small songbirds have apparent survival rates of $\sim .5$; for a stable population,
recruitment would be similar, which in lieu of direct information on vital
rates, might make the 2 parameters difficult to separate. In the American Robin
example in Dail and Madsen (2011), survival and recruitment were estimated at
reasonable values of .7 and .5 (although precision was still fairly low). Could
low detection probability have something to do with this problem? From the p
estimates in Fig. B1, detection probabilities were $< .2$ (which seems low to me
given that ovenbirds are loud and sing a lot, although I suppose it has to do
with large sample radius). In the simulations in the Dail-Madsen paper, lowest
coverage for gamma and omega was found at lowest p values (p = .25). I realize
I'm stretching a bit here, but it was a little
unsatisfying to me that the vital rates parameters could not be estimated, and
the reason for why this was so wasn't too clear. The authors suggest individual
heterogeneity might be an issue and suggest incorporating prior information or
data via integrated population models to improve vital rate estimation. However,
I wonder if some additional conjecture might be added here?

\vspace{0.5cm}
\textit{[Unsatisfying to us too!  I can work on this.  Richard, I seem to remember that 
you had done some simulations on why $p$ estimates are so low from BBS data 
(and it was exactly because of large radius).  Did you publish those results somewhere
we can cite?  If not, maybe we should not address that point here.
The additional speculations
the reviewer provides seem possible, although in our robustness testing we tried
DM simulations with $\omega = \gamma = 0.5$ and $p = 0.25$, and though the DM
estimator didn't do perfect, they weren't nearly as bad as the real data.  Maybe 
they would have been with a lower simulated $p$, but it seems more likely to me that
the real data somehow came from a different process than the DM model
assumes.

Jeff, I don't have any paper about this that we could cite. However,
it seem intuitive and even necessary that p will decrease as sample
area increases. A distance sampling observation model would be better
here. Didn't we mention that in the paper?

]}
\vspace{0.5cm}

\bibliography{improved_state_space}


\end{flushleft}
\end{spacing}
\end{document}
