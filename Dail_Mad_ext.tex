\documentclass[12pt,english]{article}
%\renewcommand*\abstractname{Summary}
\usepackage[verbose,letterpaper,hmargin={2.4cm,2.4cm},vmargin={2.4cm,2.4cm}]{geometry}
\usepackage[mathlines]{lineno} %,displaymath
\linenumbers
\usepackage{setspace}
\usepackage{amsmath}
\usepackage{amsfonts}
\usepackage{bm}
\usepackage{graphicx}
\usepackage[round,comma,authoryear]{natbib}
\usepackage{color, soul}
\usepackage{fancyvrb}
\usepackage{verbatim}
\usepackage{indentfirst}
\usepackage[format=plain,justification=raggedright,labelsep=period,font={stretch=1.9}]{caption}
%\usepackage[pdftex,colorlinks=true,citecolor=black,linkcolor=black,urlcolor=black,pdftex,pdfstartview=FitH,bookmarks=true]{hyperref}
\usepackage[pdftex,hidelinks,pdfstartview={XYZ 1 1 1.5}]{hyperref}
\usepackage{rotating}
\usepackage[compact]{titlesec}
\titleformat{\section}[display]{\normalfont\bfseries}{}{0em}{\centering}
\titleformat{\subsection}[display]{\normalfont\itshape}{}{0em}{\centering}
\bibliographystyle{ecology}
\setlength{\bibsep}{0.0pt}
\bibpunct{(}{)}{,}{a}{}{,}
%\hyphenpenalty=100000
\setlength{\jot}{-1.2ex}
\usepackage{etoolbox}
\AtBeginEnvironment{tabular}{\doublespacing}

\begin{document}
\renewcommand{\thefootnote}{\arabic{footnote}}
\title{Improved state-space models for inference about 
spatial and temporal variation in abundance from count data}

%\doublespacing
\author{Jeffrey A. Hostetler\thanks{Migratory Bird Center, Smithsonian 
Conservation Biology Institute, National Zoological Park, MRC 5503, 
Washington, DC 20013-7012}
   \and Richard B. Chandler\thanks{USGS Patuxent Wildlife Research Center 
        Laurel, MD 20708} \footnote{Current address: Warnell School of
     Forestry and Natural Resources, University of Georgia,
     \url{rchandler@warnell.uga.edu}, +001-706-542-5815}        
}
\date{} %remove date
\maketitle
\begin{spacing}{1.9}
\begin{flushleft}
\abstract{%\noindent 
Models of population dynamics %play a central role in theoretical and
%applied ecology, where they 
are frequently used for purposes such as testing 
hypotheses about density dependence and predicting species' responses 
to future environmental change or conservation actions. Fitting models
of population dynamics to field data is challenging because most data
sets are characterized by observation error, which
can inflate estimates of process variation if ignored.  
%Failure to account for observation error in such models can result in 
%bias. Therefore, ecologists have increasingly relied on 
Recently, state-space models have been developed to deal with this problem
%these probdirectly model 
by directly modeling both the observation error and the ecological
process of interest.
Conventional state-space models, however, have
several important limitations: (1) they assume that random effects are
Gaussian distributed, which implies that abundance can be negative and
that false positive observation errors are equally likely as false negative
errors; %results in unrealistic predictions
%(e.g. negative values of abundance) and an implausible mode
%and observation error are Gaussian random effects
%the parameters are not identifiable
%in many common situations, 
(2) they do not admit spatial variation in
population dynamics; and (3) some of the parameters of the model are not
estimable. % they assert an implausible model for
%we have no clear interpretation of the
%observation error. 
We demonstrate how each of these problems can be
resolved using a class of hierarchical models proposed by
\citet[Biometrics]{dail_madsen:2011} 
%for spatially-replicated time series data 
that attributes observation error to imperfect detection.
We expand this class of models to accommodate classical growth
models (e.g. exponential and Ricker-logistic), zero-inflation,
and random effects. 
% such as observer-specific detection probabilities. 
We also present methods for forecasting
population size under future environmental conditions. Implementation
of these ideas is possible using either frequentist or Bayesian
methods, as demonstrated by accompanying {\bf R} and {\bf JAGS} code. %We provide code to fit these models using the \textbf{R} package
%\texttt{unmarked} and \textbf{JAGS} is also provided.
Results of a simulation study suggest that bias is %was 
negligible and coverage
nominal for the proposed model extensions. An analysis of data from
the North American Breeding Bird Survey highlights how these methods
can be readily applied to existing data, but it also suggests 
that precision will be low when direct information about detection
probability (such as is collected using distance sampling or
replicated counts) is lacking.
}

%\vspace{0.5cm}

\textit{Key words}: abundance, Dail and Madsen model, density-dependence,
detectability, Gompertz-logistic model, immigration, open population point 
count models, population dynamics, random observer effects, Ricker-logistic model, zero-inflated

%\newpage
\vspace{0.5cm}

\section*{Introduction}
%Theoretical ecology requires models of population dynamics for testing
%hypotheses regarding spatial and temporal variation in
%abundance. For example, much theoretical work has focused on
Models of population dynamics are vital in both theoretical and
applied ecological research. For instance, %the have been used to assess
the importance and existence of phenomenon such as
density-dependent population regulation, population cycling, and
spatial synchrony have been studied by comparing data from natural
populations with theoretical models
\citep{may:1975,turchin:1990,bjornstad_etal:1999,royama:1977}. %,dennis_taper:1994
%and population models are required to evaluate associated hypotheses.
In applied contexts, population models are used for estimating extinction probabilities
\citep{schoener_spiller:1992,nadeem_lele:2011,hostetler_etal:2012} and for predicting the
effects of future environmental conditions or conservation actions on
population size \citep{jamieson_brooks:2004,hatfield_etal:2012}.

Several challenges are routinely encountered when trying to fit models
of population dynamics to data from field studies. % is that 
%In order to address these questions, two complicating factors must
%be confronted when fitting population models to data. 
First, deterministic models of population dynamics are virtually always inadequate due to process
variation, the inherent stochasticity in demographic parameters and environmental
conditions \citep{bjornstad_grenfell:2001,saether_engen:2002}.
Second, abundance---the natural state variable in studies
of population dynamics---can rarely be observed perfectly in field
studies because of observation error, such as imperfect
detection \citep{link_nichols:1994,kery_etal:2009}.

Recently developed state-space models %are a widely used approach for
have made it possible to %become the primary tool for 
study population dynamics while accounting for both process variation
and observation error \citep{devalpine_hastings:2002,
  buckland_etal:2004, dennis_etal:2006}. Classical state-space
models are time series models in which the true state of the
system (e.g. population size during each year) is modeled as a hidden
process, and the observed data are modeled conditional on this process
and the observation error. One reason for the widespread adoption of
state-space models in ecology is that failure to account for 
%process variation and 
observation error can bias estimators of abundance and
population growth parameters. For instance, the strength of
density dependence will be overestimated if observation error is
ignored \citep{link_nichols:1994,shenk_etal:1998}.

A simple state-space model can be described as follows.
Let $N_t$ be the abundance of a species during year $t$, for
$t=1,\hdots,T$, and let $X_t$ be
the observed data, which differs from $N_t$ due to observation error,
a random effect denoted $\zeta_t$. Temporal variation in $N_t$ is
modeled using a population growth model, $\mu(N_{t-1})$
and random process variation denoted $\eta_t$.
% Richard, is switching this from $\grave{o}$ to $\eta$ okay?  Andy thought
% the o with accent was weird and I think I agree.  $\nu$ seems standard,
% but we use it for a slightly different meaning (ES only) later.  
The growth model may be density-dependent, as in the case of the 
Ricker-logistic model, or it might be density-independent, such as when growth
is exponential.
%, in which $\mu(N_{t-1}) = N_{t-1}e^{r}$ where $r$ is the intrinsic rate of increase.
A generic model can be written: %The full model can now be written as:
\begin{linenomath*}
\begin{gather}
  \label{eq:ss1}
  \begin{align}
    N_1 &= X_1 \nonumber \\
N_t &= \mu(N_{t-1}) + \eta_{t-1} \quad \text{for} \;
t=2,\hdots,T  \\
X_t &= N_t + \zeta_t \qquad \qquad \;\, \text{for} \;
t=1,\hdots,T. \nonumber 
  \end{align}
\end{gather}
\end{linenomath*}
%where $\eta_t$ is the random effect allowing for process
%variation unaccounted for by the deterministic model. 
In classical
state-space models, the two sets of random effects
are assumed to be i.i.d. Gaussian deviates:
$\eta_t \sim \mathrm{N}(0, \sigma)$ and
$\zeta_t \sim \mathrm{N}(0, \tau)$. %Standard practice is to
%ignore 
The process variation associated with $N_1$ is often ignored, or
%, as indicated by Eq~\ref{eq:ss1a}. Alternatively, 
the population is assumed to be at equilibrium such that the $N_1$ can
be regarded as an outcome of the equilibrium distribution.

Even though state-space models such as that shown in Eq.~\ref{eq:ss1}
overcome many of the limitations associated with the application of
classical models of population dynamics, 
%are among the most widespread approaches for modeling population dynamics
%using time series data, 
several problems exist. %, which we explain in the following paragraphs. 
% First,
% time series data and the underlying abundance parameters of interest
% are typically integer-valued, as in the case of count data, raising
% concerns about the Gaussian assumption for the random
% effects. Second, the model for observation error is implausible. %has little biological basis. 
% Third, the model does not account for spatial variation in abundance. Lastly, some of the
% parameters of the model are not estimable. We briefly discuss each of
% these points before describing a general approach to resolving these
% shortcomings.
%Use of the Gaussian distribution for modeling random process variation
%and observation error is often motivated by convenience rather than
%biology. Specifically, t
The assumption that random effects are Gaussian distributed makes it
easier to estimate parameters because methods such as the Kalman
filter can be applied %The Gaussian assumption allows for efficient methods of parameter
%estimation, such as the Kalman filter
\citep{dennis_etal:2006}; however, % which are not available otherwise % which is much more computationally efficient
%than estimation methods when random effects are not Gaussian distributed
%\citep{devalpine_hastings:2002}. %The problem with this is that it
%However, 
these models effectively assume that abundance can be take on any
real value, and hence they can predict negative values of abundance. %including negative values, 
%This approach allows for negative and non-integer values of the state-variable, which is
%inconsistent with the observed data, and may 
%and these models can result in implausible predictions.
Transforming the count data and the underlying abundance variables 
do little to resolve these issues \citep{ohara_kotze:2010}.

With respect to the observation error, the Gaussian assumption
%Another problem with standard state-space models is that the model for
%the mean zero Gaussian distribution for the observation error
implies that false positive detections are
equally likely as false negatives, a pattern inconsistent with most
findings \citep{miller_etal:2011}. 
%. To our knowledge, all studies to
%date have shown that false negative detections are much more common
%that false positives \citep{miller_etal:2011}.
%will be
%higher than $N_t$ as often as it is lower than $N_t$. 
%Identifying a biological mechanism that would cause such symmetric errors
%is difficult.
A more likely form of observation error, and one that has been recognized for well
over a century, results from failing to detect individuals that are
present. Imperfect detection may be attributable to
characteristics of the species under study, such as its elusiveness,
or to individuals being missed by the ecologist collecting the data in the field.
Although a vast number of methods have been devised for accounting for
this form of observation error, rarely have these methods been
integrated into state-space models of population dynamics \citep[but
see][]{buckland_etal:2004}.

Other limitations of state-space models, as commonly applied in ecology,
include the failure to admit spatial variation in abundance and
the fact that some parameters are not identifiable
\citep{polansky_etal:2009}. 
% without making 
%Another limitation of many state-space models is that they do 
%Common formulations of state space models do 
%not admit spatial variation in abundance or demographic
%parameters. 
Ignorning spatial variation simply reduces the scope of the
inferences, 
% that can 
%be drawn from the models, it disregards the fact that space is an
%important factor in the regulation of natural populations \citep{murdoch:1994,
%pulliam:1988, gill_etal:2001}.
but identifiability problems require %lack of identifiability because
model constraints or the assertion of additional assumptions. 
%A more serious problem %than the ones associated with the Gaussian assumptions 
%is that the parameters of simple state-space models
%such as Eq~\ref{eq:ss1} are not identifiable in many
%circumstances \citep{polansky_etal:2009}. 
Specifically, abundance in
the first year cannot be estimated because there is only one data
point (the count in the first year) that carries any information about
it. To remedy this, pratitioners either assume that there is no
observation error in the first year, or they assume the population is
at equilibrium, which may defeat the purpose of many studies of
population dynamics. 
%Recall that the
%Eq.~\ref{eq:ss1} did not specify distributions for $N_1$, % or $X_1$;
%although $N_1$ is a random variable. Thus, uncertainty exists
%that should be accounted for in the model.  
%%are needed to fully specify the model.
%To be more specific, a fully-specified state-space model requires at least
%three probability distributions, which we represent
%using bracket notation (i.e., $[Y|\Omega]$ is interpreted
%as the probability distribution of the random variable $Y$ given the
%parameter $\Omega$). The three probability distributions required for a
%state-space model are:
%%%% JEFF, does this extra sentence help? This notation comes from
%%%% several of the original state-space papers and is widespread in
%%%% books like Andy's and Bill's.
%\hl{[I'm not familiar with bracket notation,
% so these equations are confusing to me. How widely used is this
% notation in ecology? Will others likely understand this? Also, we might want
%to save space by trimming a bunch of this discussion back down.]}
%\begin{subequations}
%  \label{eq:ss2}
%  \begin{align}
%    [N_1&|\bm{\theta}] \label{eq:ss2a} \\ \label{eq:ss2b}
%[N_t&|N_{t-1},\bm{\Theta}] \quad \text{for} \; t=2,\hdots,T \\
%\label{eq:ss2c}
%    [X_t&|N_t,\bm{p}]  \qquad \; \text{for} \; t=1,\hdots,T
%  \end{align}
%\end{subequations}
%where $\bm{\theta}$ are the process variation parameters for the
%initial state, $\bm{\Theta}$ are the process variation parameters
%influencing how abundance changes over time, and $\bm{p}$ are the
%observation error parameters. As mentioned previously, Eq.~\ref{eq:ss2b}
%and Eq.~\ref{eq:ss2c} are assumed to be Gaussian in classical state-space
%models, but what should be the distribution for Eq.~\ref{eq:ss2a}?
%The three lines of general model
%described by Eq.~\ref{eq:ss2} correspond to the three equations shown
%for the specific example in Eq~\ref{eq:ss1}. However, in this case,
%we allow $N_1$ to be a random variable. But what should its
%distribution, $[N_1|\bm{\theta}]$ be?
%And how can $\bm{\theta}$ be estimated given that there is only a single observation
%vailable? 
%Often, these problems are simply assumed away. For
%example, sometimes researchers assume that there is no process variation in
%the first year, although this seems unjustified. Alternatively, the
%population may be assumed to be at equilibrium such that the expected
%value of $N_t$ is constant through time. Although this 
%makes the parameters $\bm{\theta}$ identifiable, assuming equilibrium
%defeats the objective of many studies of population dynamics, namely
%determining why a population varies over time.
%% OR
%Sometimes researchers assume that there is no process variation in 
%the first year or that the population is at equilibrium such that the expected
%value of $N_t$ is constant through time, but these approaches are unsatisfactory.

Several extensions of state-space models have been proposed to
overcome the limitations described above. \citet{devalpine_hastings:2002} and
\citet{brooks_etal:2004} described methods for fitting models with non-Gaussian
distributions for the process and observation errors. Observation models with
more intuitive interpretations, such as those that explicitly model
detection probability, have been proposed by 
\citet{kery_etal:2009}. \citet{lele_etal:1998} and 
\citet{kery_etal:2009} developed models allowing for inference about
spatial and temporal variation in abundance, and their developments
also resolved the problems of non-identifiability for the parameters
of the initial state at time $t=1$. Of these extensions, only
the work by \citet{kery_etal:2009} addressed several of these limitations
simultaneously; however, their model did not include serial
dependence, which is a hallmark of population models. %This
%limits the utility of their model for making inferences about explicit
%population processes.

In this paper, we focus on the model of \citet[henceforth the DM model]{dail_madsen:2011}
that resolves each of the problems outlined above. 
%simultaneous resolves each of the aforementioned problems with
%traditional state-space models, and is designed for simple count data.
%In the following section, we describe the DM model in its original
%form and explain how it resolves each of the deficiencies with
%standard state-space models. 
%In Section~\ref{sec:ext}, 
Our aim is to extend the
model to accommodate classical models of population growth and
to handle several features common to ecological
time series, such as excess zeros and unexplained random variation. 
%Specifically, we describe methods for accommodating
%excess zeros and nuisance variables such as random observer effects.
Both frequentist and Bayesian methods of inference are discussed, and
we evaluate the performance of the model using a simulation study and by
analyzing data from the North American Breeding Bird Survey (BBS), one of
the most spatially and temporally extensive sets of count data on
vertebrate populations \citep{robbins_etal:1986}.
%code for fitting models is presented in the appendices.
%The Bayesian method
%is attractive in that it can accommodate prior information about
%detection parameters, which is available for many species, yet not
%collected as part of many existing monitoring programs.
%In Section~\ref{sec:app}, we evaluate the performance of the model
%extensions using a simulation study and by analyzing data from the
%North American Breeding Bird Survey (BBS), one of
%the most spatially and temporally extensive sets of count data on
%vertebrate populations \citep{robbins_etal:1986}. %The overarching aim
%of the paper is to provide ecologists with flexible and accessible means of
%addressing important questions related to the variation of abundance in
%space and time. 
%believe these extensions will increase the utility of the model for
%application to existing datasets such the Breeding Bird Survey.


\section*{The Dail-Madsen Model}
\label{sec:dm}

The DM model is an extension of the 
\citet{royle:2004biom} $N$-mixture model, which allows for inference about spatial
variation in abundance when individuals cannot be detected with
certainty. To estimate parameters associated with both abundance 
and the detection process, the original $N$-mixture model uses replicate
observations at each site. The observations are collected during sufficiently
short time intervals such that the population can safely be
assumed to be closed with respect to births, deaths, and movement. The DM
model relaxes this closure assumption and includes explicit parameters
describing population change over time.


%\subsection{The Data}

The DM model requires count data collected at $R$ sites, each of
which is surveyed on $T$ primary sampling periods. 
%A site is ideally a
%well-defined region of a study area, such as a wetland or a
%patch of early-successional habitat, although it could be an arbitrarily
%defined region such as a randomly located survey plots. The time
%frame of the study is arbitrary, but in general
%it will be sufficiently long to allow for temporal variation in
%abundance. For instance, a primary sampling period could be a
%one month long breeding season, which could be surveyed once per
%year for $T$ years. If the a site is surveyed on $J$ occasions during a
%primary period, these are called secondary sampling occasions.
%In the case where no secondary sampling was conducted, i.e. $J=1$, 
Let $X_{i,t}: i=1,\hdots,R; t=1,\hdots,T$ denote the count data
at site $i$ and primary period $t$. 
%If secondary sampling periods were
%used, an extra dimension is added so that we have $X_{ijt}:
%j=1,\hdots,J$. 
In general, $X_{i,t}$ will be lower than abundance, 
%actual quantity of interest, abundance,
%denoted 
$N_{i,t}$, but in cases where detection is perfect,
abundance is observed directly such that $X_{i,t} \equiv N_{i,t}$.

%\subsection{The Original Formulation of the Model}

Like traditional state-space models, the DM model includes 
three conditionally related processes corresponding to: 
% shown in Eq.~\ref{eq:ss1}, which correspond to 
(1) initial abundance, i.e. the
abundance at site $i$ during the first primary period,
denoted (2) abundance at time $t$ (for $t>1$) which depends upon
abundance at $t-1$, and (3) the
detection process \citep{dail_madsen:2011}.
The first two processes describe the state process---the
variation in abundance in space and time. The third process %simply 
describes the the relationship between
abundance and the observed count data.

\subsection*{Initial abundance}

%Recall that 
Conventional state-space models assume that the
distribution for the initial time period %,
%$[N_1 | \bm{\theta}]$, 
is either the equilibrium
distribution, or has zero variance.
In contrast, \citet{dail_madsen:2011} proposed modeling $N_{i1}$
as either a Poisson or negative binomial random variable:
\begin{linenomath*}
\begin{gather}
N_{i,1} \sim \mathrm{Pois}(\Lambda) \nonumber \\
\text{or} \nonumber \\
N_{i,1} \sim \mathrm{NB}(\Lambda, \alpha)
\label{eq:N1}
\end{gather}
\end{linenomath*}
where $\Lambda_i$ is the expected abundance at site $i$ during
year 1.
The Poisson distribution assumes that the mean of $N_{i,1}$ is
equal to its variance, whereas the negative binomial distribution allows the
variance to be greater than the mean with the amount of
overdispersion determined by the parameter $\alpha$.
%In the Poisson case, the equivalent expression of
%$[N_1 | \bm{\theta}]$ (Eq.~\ref{eq:ss2a}) is
%$[N_{i1} | \Lambda] =\frac{\Lambda^{N_{i1}}e^{-\Lambda}}{N_{i1}!}$.
%\begin{equation}
%[N_{i1} | \Lambda] =
%\frac{\Lambda^{N_{i1}}e^{-\Lambda}}{N_{i1}!}.
%  \label{eq:N1br}
%\end{equation}
% Maybe we don't need the Poisson distriubtion equation? - Jeff

Regardless of the specified distribution, the model for initial
abundance has two distinguishing features. First, it provides a
mechanism for characterizing spatial variation in abundance. For
instance, one might consider the influence of some environmental
covariate ($x_i$) on abundance using a log-linear
model such as $\log(\Lambda_i) = \beta^{\Lambda}_0 +
\beta^{\Lambda}_1
x_{i}$. The second important point is that the spatial
replicates resolve the
problem of parameter non-identifiability that are common to
standard state-space models because, as demonstrated by
\citet{royle:2004biom},
process variation and observation error can be estimated from
spatially-replicated count data. Hence, the first component of
the model addresses both the issues of spatial inference and
parameter identifiability discussed previously.

\subsection*{Abundance in subsequent time periods}

The DM model assumes that abundance in time $t$ is a function of
abundance at time $t-1$, i.e. abundance at each site evolves as a
first order Markov process. %, although higher order processes are also possible.
\citet{dail_madsen:2011} considered several models to describe the temporal dynamics;
however, in each case they modeled $N_t$ as the sum of two random variables:
$S_{i,t}$, the number of indivduals surviving from $t-1$ and not
emigrating; and $G_{i,t}$ the number of new individuals entering
the population. Their most general model was
\begin{linenomath*}
\begin{equation}
\left.\begin{aligned}
S_{i,t}|N_{i,t-1} &\sim \mathrm{Bin}(N_{i,t-1}, \omega) \\
G_{i,t}|N_{i,t-1} &\sim \mathrm{Pois}(\gamma(N_{i,t-1})) \\
N_{i,t} &= S_{i,t}+G_{i,t}
\end{aligned}\right\} \quad \text{for} \; t=2,\hdots,T
\label{eq:Nt}
\end{equation}
\end{linenomath*}
where $\omega$ is the apparent survival probability and $\gamma$
is the recruitment rate (which can depend on $N_{i,t-1}$).
\citet{dail_madsen:2011} proposed three
models for $\gamma$: the constant model,
$G_{i,t} \sim \mathrm{Pois}(\gamma)$ where recruitment does not
depend on $N_{i,t-1}$, and which simulates a ``propagule rain'' of new
individuals; the autoregressive model, $G_{i,t} \sim
\mathrm{Pois}(\gamma(N_{i,t-1}))$, which
simulates geometric or density independent growth; and the
``no-trend'' model, $\gamma = (1-\omega)\Lambda$, which assumes the
population is at equilibrium. Covariates of
$\omega$ and $\gamma$ can be easily accommodated, for example
using logit- and log-linear models respectively.

%To make the connection between conventional state-space models
%and the DM model, we need to replace $[N_t|N_{t-1}, \bm{\Theta}]$,
%Eq.~\ref{eq:ss2b}, with an expression derived from
%Eq.~\ref{eq:Nt}. This requires summing over all possible values
%of $S$ and $G$, which is accomplished using the discrete convolution:
%\begin{equation}
%[N_{it}|N_{it-1}, \omega, \gamma] =
%  \sum_{S_{it-1}=0}^{\min(N_{it},N_{it-1})}
%\binom{N_{it-1}}{S_{it-1}}\omega^{S_{it-1}}(1-\omega)^{N_{it-1}-S_{it-1}}
%\times
%\frac{\gamma^{N_{it}-S_{it-1}}e^{-\gamma}}{(N_{it}-S_{it-1})!}  \label{eq:P}
%\end{equation}
%where here we assumed no dependence of $\gamma$ on $N_{it-1}$.
%
%In the absence of movement, $\omega$ is exactly the probability of surviving
%from year $t$ to $t-1$, and $\gamma$ is the per-capita birth rate
%(under the geometric growth model). This ability to directly estimate
%demographic parameters from simple count data is one of the DM
%model's most appearling features. However,
%in the more common scenario when immigration and emigration
%occur, $\omega$ and $\gamma$ no longer can be interpreted as vital
%rates. Rather, $\omega$ is the probability of surviving and not
%emigrating, and $\gamma$ is the sum of the birth rate and the immigration rate,
%i.e. the recruitment rate. In such cases, we propose
%replacing Eq.~\ref{eq:Nt} with standard population models.

\begin{comment}
Irrespective of the model for population dynamics, it is
worthwhile
to note that the discrete distributions used in the DM model
avoid
the issues associated with the Gaussian distribution used to
model
  process variation in standard state-space models.
\end{comment}

\subsection*{Observation process}

%Eqs.~\ref{eq:N1} and \ref{eq:Nt} fully specify the state model,
%i.e. the model for abundance dynamics. This state
%model has allowed us to replace the initial abundance
%distribution and transition models of conventional state-space models, with
%alternatives that (1) allow for modeling spatial variation in
%abundance, (2) respect the discrete nature of the counts, and
%(3) have parameters that are estimable.
%The one
%remaining task is to replace the observation model
%$X_t = N_t + \zeta_t$ (Eq.~\ref{eq:ss1c}) with one that can be
%more easily interpreted. 
The observation model adopted by \citet{dail_madsen:2011} is the same
binomial model proposed by \citet{royle:2004biom}: % assumes that individuals are missed due to imperfect
%detection. The simplest model for imperfect detection is
\begin{linenomath*}
\begin{equation}
  X_{i,t} \sim \mathrm{Bin}(N_{i,t}, p)
  \label{eq:p1}
\end{equation}
\end{linenomath*}
where $p$ is the probability of detecting each individual. Variation
in detection probability can be modeled as a function of site-specific
or occasion-specific covariates using, for example, a logit-linear model.
%or equivalently,
%\begin{equation}
%[X_{it} | N_{it}, \bm{p}] = \binom{N_{it}}{X_{it}}
%\bm{p}^{X_{it}}(1-\bm{p})^{N_{it}-X_{it}}.
%  \label{eq:1}
%\end{equation}
%Under this model, $\bm{p}$ is interpreted as the detection
%probability parameter.

%%%% Save for discussion?
%Surprisingly detection probability, $p$, can be estimated without
%secondary sampling occasions. However, this is only possible
%if the parametric assumptions of the population dynamics are met
%exactly. In practice, it is preferable to obtain direct information
%about $p$ using secondary sampling occasions or some other auxiliary
%data, \hl{as is demonstrated in section....}

%The entire model can now be written:
%\begin{gather}
%  N_{i1} \sim \text{Pois}(\Lambda) \nonumber \\
%S_{it}|N_{it-1} \sim \mathrm{Bin}(N_{it-1}, \omega) \nonumber \\  
%G_{it}|N_{it-1}\sim \mathrm{Pois}(\gamma(N_{it-1})) \\
%  N_{it} = S_{it}+G_{it} \nonumber \\
%  X_{it} \sim \text{Bin}(N_{it}, p) \nonumber
%  \label{eq:hm}
%\end{gather}

%\subsection{Statistical Inference}

%\subsubsection{Maximum likelihood}

%Maximum likelihood estimation of parameters in models with
%discrete random effects requires removing them by
%summation. For Markovian models, this can be done recursively,
%making the likelihood tractable \citep{dail_madsen:2011}. 
%A general expression of the likelihood can be written:
%\begin{multline}
%  \mathcal{L}(\Lambda, \bm{\Theta}, p | \{X_{it}\}) = \\
%  \prod_{i=1}^R \left\{ \sum_{N_{i1}=0}^{\infty}
%  [X_{i1}|N_{i1}, p][N_{i1}|\Lambda] \left\{
%%  \prod_{t=T}^2 \left\{
%  \sum_{N_{i2}=0}^{\infty} \dotsm \sum_{N_{iT}=0}^\infty
%  [X_{it}|N_{it}, p][N_{it}|N_{it-1}, \bm{\Theta}] %\right\}
%  \right\} \right\}
%  \label{eq:like}
%\end{multline}
%where $\infty$ is replaced by a finite upper bound $N_{max}$, which
%%% Can we rename the upper bound?  We're also using K
%%% for equilibrium abundance.
%should be high enough such that the MLEs are not affected by increasing
%it further. 
%The likelihood can be maximized numerically using the
%\textbf{R} package \texttt{unmarked} \citep{fiske_chandler:2011}. Examples
%are given in the Appendix.



\section*{Model Extensions}
\label{sec:ext}

\subsection*{Population growth models}

% Preliminary DM model runs for several species tended to lead to
% estimates of survival that were unrealistically high and recruitment
% that were unrealistically low, or the reverse (compared to independent
% demographic analyses). The DM models are able to partition changes in
% abundance to survival and recruitment in part by making strong
% distributional assumptions. % (Dail and Madsen 2011).
% When those assumptions are heavily violated the models may proportion
% population growth incorrectly into survival and recruitment, even if
% they estimates population growth accurately.

Partitioning population growth into survival and recruitment
is ideal in terms of providing a mechanistic description of population
dynamics, but this is not always possible using simple count data,
especially when the sites are not closed with respect to immigration
and emigration. %Furthermore, a
%simpler model would have several merits: faster running time, fewer
%total model combinations, and possibly more realistic estimates.
When the mechanistic model is unrealistic, we suggest replacing it
with classical population growth models. %that have a long history in ecology.
The exponential growth model is the simplest: 
$N_{i,t} = N_{i,t-1}e^r$ where $r$ is the intrinsic
rate of increase. %Although demographic and environmental stochasticity
%in $r$ could be modeled as functions of covariates or random effects, a
%simpler way of 
To allow for demographic stochasticity, we can regard $N_{i,t}$ as a Poisson 
random variable, simplifying Eq.~\ref{eq:Nt} to:
\begin{linenomath*}
\begin{equation}
  N_{i,t} \sim \mathrm{Pois}(\exp(r)N_{i,t-1}).
\label{eq:exp}
\end{equation}
\end{linenomath*}

%This,
%like the geometric-recruitment version of the model, is a variant on a simple
%density-independent, exponential model of population growth.
Density-dependent versions of the model are also possible.  For
example:
\begin{linenomath*}
\begin{equation}
  N_{i,t} \sim \mathrm{Pois}(N_{i,t-1}\exp(r(1-N_{i,t-1}/K)))
\label{eq:rick}
\end{equation}
\end{linenomath*}
where $K$ is the stable equilibrium of the population and $r$ is
the instantaneous population growth rate at low population
densities, and both parameters are constrained to be positive. This is
a stochastic version of the Ricker-logistic population growth model
\citep{ricker:1954}. Another option is a modified Gompertz-logistic
density-dependent model \citep{hart_gotelli:2011}:
\begin{linenomath*}
\begin{equation}
N_{i,t} \sim \mathrm{Pois}(N_{i,t-1}\exp(r(1-\log(N_{i,t-1}+1)/\log(K+1))))
\label{eq:gomp}
\end{equation}
\end{linenomath*}
Here the interpretations of $r$ and $K$ are similar to in the
Ricker-logistic model. Because a single Poisson distribution controls
the distribution of $N_{i,t}$ in each of these models, the discrete 
convolution used by \citet{dail_madsen:2011} to construct the
likelihood is not required, speeding up processing time.


\subsection*{Immigration models}

One limitation of the geometric-recruitment, exponential,
Ricker-logistic, and Gompertz-logistic versions of the DM models all
%share a common feature (or bug): once the 
is that they include no mechanism for a local population to recover
after extinction. %at a site reaches 0, it must remain at 0
%because all contributions to population growth are assumed to be
%local.  
However, these models can be generalized 
%We therefore generalized these models 
to include both internal and external (immigration) contributions
to population growth. For example, an exponential plus immigration
model is:
\begin{linenomath*}
\begin{equation}
  N_{i,t} \sim \mathrm{Pois}(\exp(r)N_{i,t-1} + \iota)
  \label{eq:expimm2}
\end{equation}
\end{linenomath*}
where $\iota$ represents the average number of immigrants per year, and is
constrained to be
positive (this is equivalent to separate Poisson processes for
growth and immigration).  %This model is close to
%the constant-recruitment DM model (Eq.~\ref{eq:Nt}), with $\exp(r)$ instead
%of $\omega$ and $\iota$ instead of $\gamma$, except that the first
%process is Poisson distributed instead of binomial. 
The geometric-recruitment, Ricker-logistic,
and Gompertz-logistic models can be extended to allow for immigration in the
same way.   Deterministic versions of the Ricker-logistic + immigration and the 
Gompertz-logistic + immigration do reach stable equilibriums, but the equilibriums 
are not at $K$ and cannot be solved for analytically \citep{otto_day:2007}.  Therefore, we
refer to $K$ in these models as the semi-equilibrium abundance.
\begin{comment}
Similarly, the Ricker-logistic + immigration model can be represented as:
the
Gompertz-logistic + immigration model as: (13) and the geometric-recruitment +  immigration model as: (14)
\end{comment}

% Save for discussion
\begin{comment}
These immigration models are based on the assumption that number
of
immigrant is not dependent on local density; however, extendingthese models in this way is conceptually straight-forward, a
Other
  versions of these models with different assumptions are
possible. The Ricker-logistic + immigration and Gompertz-logistic + immigration
models
can be justified by the further assumption that immigrants
arrive as
  adults immediately before the point-count survey (so that
density-dependent processes do not have a chance to act on thenumbers of immigrants; Otto and Day 2007 p. 161).  Deterministicversions of these models do reach stable equilibriums, but theequilibriums are not at K and cannot be solved for analytically  (Otto and Day 2007).
\end{comment}

%We have implemented all preceding models in
%a maximum likelihood framework by extending the
%\texttt{unmarked} package
%\citep{fiske_chandler:2011} in \textbf{R} \citep{R-2012}.


%\subsection{Range model of initial abundance}
\subsection*{Excess zeros}

%\citet{dail_madsen:2011} suggested two distributions for modeling
%initial abundance: Poisson and negative binomial. We have
%extended their model to include another distribution, the zero-inflated
%Poisson. This could be useful when, for example, one is modeling
%the abundance of several species with the same set of point
%count surveys for all species, but some survey sites are
%outside the range of some of the species. 
The negative binomial distribution used in $N$-mixture models can have
a very long right tail, and its mean to variance ratio may not be
ideal in all cases. An alternative is to consider zero-inflated
distributions, such as the zero-inflated Poisson:
%The distribution
%of initial abundances can be represented as:
\begin{linenomath*}
\begin{equation}
N_{i1} \sim \left\{
\begin{aligned}
\mbox{Pois}(0) &\; \text{with probability} \; \psi \\
\mbox{Pois}(\Lambda) &\; \text{with probability} \; (1-\psi)
\end{aligned} \right.
\label{eq:ZIP}
\end{equation}
\end{linenomath*}
where $\psi$ represents the proportion of extra zeros. 
Mechanistically, this distribution might be appropriate when not all sites are 
within the range of a species.

%These models allow three sources of zero counts by observers: a
%species was at a site but not detected; the site was within the
%species' range but no individuals occured at that site in that
%year; and the site was outside the species' range. Furthermore,
Under this formulation, 
detection, abundance, and zero-inflation can be modeled separately as
functions of covariates. For example, detection of a
species might depend on wind speed,
abundance on forest type and weather, and zero-inflation upon elevation and
climate.  
%This approach combines elements of occupancy modeling
%\citep{mackenzie_etal:2006} and abundance modeling.
%, and could prove useful
%for distinguishing species' fundamental and realized niches
%(Hutchinson 1957).

%The zero inflation factor can also affect
%recruitment or population dynamics (see below).
%We also considered a zero-inflated negative binomial model of initial
%abundance, but this model did not perform well in preliminary tests.

%\subsection{Dynamic range models}

The zero-inflated Poisson distribution can be applied to recruitment and 
population growth terms as well as initial abundance. For example, 
the recruitment term of the constant-recruitment DM model
(Eq.~\ref{eq:Nt}) can be modified as follows:
\begin{linenomath*}
\begin{equation}
G_{i,t} \sim \left\{
\begin{aligned}
\mathrm{Pois}(0) &\; \text{with probability} \; \psi \\
\mathrm{Pois}(\gamma) &\; \text{with probability} \; (1-\psi)\end{aligned} \right.
\label{eq:ZIPts}
\end{equation}
\end{linenomath*}
%We have implemented
%zero-inflated dynamics in the Bayesian framework using program
%\textbf{JAGS} \citep[version 3.2.0]{plummer:2003}.
%(version 3.2.0; Plummer 2003)


\subsection*{Environmental and demographic stochasticity}
\begin{comment}
Even though the DM model is conceptually simple, %even in
%its most stripped down form 
it includes many random effects (the $N$'s)
making it challenging to compute the likelihood or implement MCMC
algorithms. Nonetheless, in some cases additional random effects
may be of interest. For example, in conventional state-space models,
environmental stochasticity is often modeled as $r_t \sim
\mathrm{Norm}(\mu_r, \sigma_r)$.
\end{comment}
The original DM models allows for demographic stochasticity --- variation in population growth 
rate due to the randomness of birth and death processes --- %through random 
via the binomial and Poisson distributions for apparent survival and recruitment.  Our population
growth extensions use a single Poisson distribution for abundance. %: $N_{i,t} \sim 
%\mathrm{Pois}(f(N_{i,t-1}))$.  
% JEFF: I'm not sure what you're driving at here. Do these papers
% assume a Poisson distribution too?
This is also a commonly used and mathematically equivalent 
method of modeling demographic stochasticity 
\citep{bonsall_hastings:2004,melbourne_hastings:2008}.  

One way to model environmental stochasticity --- variation in population growth
rate due to stochastic environmental conditions --- is by explicitly
modeling dynamic parameters as functions of environmental covariates, such
as weather.  For example, in the Gompertz-logistic model (Eq.~\ref{eq:gomp}), $r$ could be a 
function of temperature ($C$) and $K$ a function of rainfall ($M$):
\begin{linenomath*}
\begin{gather}
\log(r_{i,t}) = \beta^{r}_{0} + \beta^{r}_{1}C_{i,t} \nonumber \\
\log(K_{i,t}) = \beta^{K}_{0} + \beta^{K}_{1}M_{i,t}  
%N_{it} \sim
%\text{Pois}(N_{it-1}\exp(r_{it-1}(1-\log(N_{it-1}+1)/\log(K_{it-1}+1)))) \nonumber
\label{eq:weather}
\end{gather}
\end{linenomath*}

% JEFF: It seems to me that this approach could be used to account for
% either demographic or enviro stochasticity. How can you know what
% the variation in attributable to?
Another way to model environmental stochasticity is as lognormal variation in observed local population 
growth rate \citep{bjornstad:2001,bonsall_hastings:2004}.  This could be applied to both density-independent and density-dependent models.
For example, for the Ricker-logistic + immigration model:
\begin{linenomath*}
\begin{equation}
N_{i,t} \sim
\mathrm{Pois}(N_{i,t-1}\exp(\nu_{i,t} + r(1-N_{i,t-1}/K)) + \iota)
\label{eq:nuRand}
\end{equation}
\end{linenomath*}
where $\nu$ is a variable with a normal distribution, mean value of 0, and standard deviation $\sigma_\nu$.  
Three variations suggest themselves.  The first has independent environmental stochasticity 
between sites, and might be appropriate where sites are widely dispersed.  In the second, 
all sites have the same environmental conditions in a given year (regional stochasticity, $\nu_{i,t} = \nu_{t}$),
which might be appropriate where the geographic or environmental range is small \citep{hanski:1998}.  
Third, where sites can be close together but the range of sites is broad, a multivariate normal distribution 
for $\nu_{i,t}$ with distance-dependent correlation coefficients %(geographic or otherwise) 
between sites in a year might be appropriate.

\subsection*{Random observer effects}

Variation in detection probability among observers is another
source of random variation that may need to be considered when applying
the DM model. For example, differences in observers' ability to see,
hear, or identify birds has long been recognized as a potential source of error
in avian point count surveys such as the BBS 
\citep{robbins_etal:1986,sauer_etal:1994auk,campbell_francis:2011}.%,alldredge_etal:2007auk %diefenbach_etal:2003,
%Estimating a separate detection probability for each observer
%can be difficult and reduces one's ability to estimate the quantities
%of interest.  This problem is compounded by the fact that
%observers differ greatly in the number of surveys they have
%run (so that many observers' separate detection probabilities
%could not be accurately estimated).

Current BBS trend estimators deal with this problem by
treating observer identity as a random %(as opposed to a fixed)
effect \citep{link_sauer:2002,sauer_link:2011}.
%(Link and Sauer 2002, Sauer and Link 2011).
%This allows observer-specific differences in detection probability,
%but assumes that observers are selected at random from a pool
%of potential observers.  Models that contain both random and
%fixed effects are referred to as mixed models.  Often in mixed
%models (as in this case), the random effect is modeled not
%because it of interest in itself, but to avoid bias in the
%estimates of the fixed effects.
%
To include random observer effects in DM models, 
% and these extensions, 
Eq.~\ref{eq:p1} can be modified to:
\begin{linenomath*}
\begin{gather}
X_{i,j,t} \sim \mathrm{Bin}(N_{i,t}, p_j) \nonumber \\
\mathrm{logit}(p_j) \sim \mathrm{N}(\mu_p, \sigma_p)
\label{eq:pobs}
\end{gather}
\end{linenomath*}
where $X_{i,j,t}$ is the number of individuals counted at site $i$ by
observer $j$ in year $t$, $p_j$ is observer-specific detection probability,
$\mu_p$ is the mean detection probability (on the logit scale), and $\sigma_p$ is
the standard deviation of the random observer effects (also on the logit scale). 
%We have implemented random observer effects in the Bayesian
%framework using program \textbf{JAGS} \citep[version 3.2.0, see Appendix]{plummer:2003}.
% (version 3.2.0; Plummer 2003)
%with the \textbf{R} package \texttt{rjags} \citep{plummer:2011,R-2012} interface.



%\subsection{Forecasting Future Population Size}

%State-space models allow for inference about future population size,
%a critical component of population viability analysis (PVA) as well
%as conservation planning. 
%Let $N^+_{t}$ denote population size at some
%future point in time and $[N^+_t]$ denote the probability
%distribution for this random variable. If we can specify this
%distribution, we can make statements about the
%probability that $N^+_t = k$ where $k$ is any possible
%population size including zero.

%The classical approach to computing $[N^+_t]$ is based upon
%Empirical Bayes methods [CITE]. 
%Note that neither Empiriis not actually a
%Bayesian method, it is used to estim.
%We take the parameter estimates and plug them back in.... For
%example to obtain the posterior distribution for $[N_{i1}]$
%\begin{equation}
%[\hat{N}_{i1}] = \frac{[N_{i1}|\hat{\Lambda}][X_{i1}|N_{i1},
%\hat{p}]}{\sum_{N_{i1}=0}^K
%[N_{i1}|\hat{\Lambda}][X_{i1}|N_{i1},\hat{p}]}
%  \label{eq:eb1}
%\end{equation}
%Once we have estimated the posterior distribution for the first
%time period, we can then proceed to estimate the posterior
%distribution for subsequent time periods using recursion.
%\begin{equation}
%  [\hat{N}_{it}] =
%  \frac{[N_{it}|N_{it-1},\hat{\bm{\Theta}}][X_{it}|N_{it},
%\hat{p}]}{\sum_{N_{it}=0}^K [N_{it}|N_{it-1},
%\hat{\Theta}][X_{it}|N_{it},\hat{p}]}
%  \label{eq:eb2}
%\end{equation}
%
%Note that when $X_{it}$ is not observed, including for future
%time periods, the distribution $[X_{it}|N_{it}]$ is taken to be
%uniform such that it has no influence on the posterior. In other words,
%the posterior for years with no observed data is determined entirely
%by the estimated posterior from the previous year and the
%population dynamics parameters.
%
%If covariates of the dynamics parameters are present, the problem
%becomes much harder because, to fully account for uncertainty,
%we need a model for the future values of the covariates as well. When
%the covariates are continuous, this requires integrating over their
%possible values, which becomes very challenging
%computationally. Indeed, simulation methods such as MCMC are the
%only viable options when numerous covariates are present, which will
%often be the case in many ecological studies.


%% COULD INCLUDE THIS PARAGRAPH, OR NOT.
%A standard approach to PVA uses previously derived estimates of current 
%population size, vital rate or population growth parameters, and environmental
%stochasticity to stochastically project the population forward and calculate probabilities
%of extinction or other measures of risk.  Estimates from DM models and the extensions
%given here can be used in this context.  Uncertainty in extinction or quasi-extinction probabilities
%can be estimated by either drawing values for parameters from their posterior distributions
%(if estimated using MCMC) or randomly drawing values for parameters using their means and standard
%errors (if estimated using maximum likelihood), and running a number of simulations with each
%set of parameter values \citep{white:2000,hostetler_etal:2012}.

%Relative to the computational challenges associated with the
%empirical Bayes methods, 
%A fully Bayesian approach to PVA is straightforward to
%implement with MCMC. Because population size at any future point %Furthermore, b
%in time is regarded as a standard, unknown parameter, computing the
%posterior distribution of $N^+_t$ requires no additional steps
%in the MCMC algorithm, other than formally stating that the count data are
%``missing'' in future years. An example of this is provided in the Appendix.



\subsection*{Statistical inference}

\citet{dail_madsen:2011} used likelihood based methods for estimating
the parameters of their model. The same likelihood functions can be
used to accomodate the alternative population growth, immigration, and 
zero-inflation (initial abundance only) models that we
have proposed. Methods for implementing these models using maximum 
likelihood estimation are included in the %. The likelihood can be maximized numerically using the
\textbf{R} package \texttt{unmarked} \citep{fiske_chandler:2011}. Examples
are given in Appendix 1.

Bayesian inference is an alternative to classical inference with
several appealing features. First, it allows direct probability
statements to be made about a hypothesis given data
\citep{link_barker:2010}.  Second, Bayesian methods offer
straight-forward approaches for combining data from multiple sources
or use existing estimates of parameters as prior distributions. Third,
one common usage of state-space models is for predicting future
population size, and this task is easily accomplished using Bayesian
methods.

In
practice, simulation methods such as Markov chain Monte Carlo (MCMC)
are used to simulate the posterior distributions. While time
consuming, MCMC may be the only viable approach for estimating
parameters, especially for hierarchical models with many random
effects.  
% Although MCMC can be slow, our experience is that it
%may actually converge faster than the amount of time required to
%maximize the likelihood shown in Eq.\ref{eq:like}, especially
%when the upper bound of the summation must be high, e.g. $N_{max}>200$.
%Furthermore, 
%MCMC may be the only possible option for estimating parameters
%using some of the extensions described below, such as allowing for
%additional random effects.
%Writing custom MCMC algorthims is often tedious and foreign to
%ecologists, but 
Software packages such as %\textbf{WinBUGS} and
\textbf{JAGS} \citep{plummer:2003} make these methods readily
available to ecologists, and we have provided examples in 
Appendix 1. 
%overcome the technical 
%difficulties by allowing users to specify the model using a simple symbolic 
%descriptions. Examples are given in the Appendix.

%For example here is \textbf{BUGS} code for the ``geometric'' model described
%in Eq.~\ref{eq:hm}:
%\begin{verbatim}
%model {
%lambda ~ dunif(0, 5)
%omega ~ dunif(0,1)
%gamma ~ dunif(0, 10)
%p ~ dunif(0,1)
%for(i in 1:nSites) {
%  N[i,1] ~ dpois(lambda)
%  y[i,1] ~ dbin(p, N[i,1])
%  for(t in 2:nYears) {
%    S[i,t-1] ~ dbin(omega, N[i,t-1])
%    G[i,t-1] ~ dpois(gamma*N[i,t-1])
%    N[i,t] <- S[i,t-1] + G[i,t-1]
%    y[i,t] ~ dbin(p, N[i,t])
%    }
%  }
%}
%\end{verbatim}




\section*{Applications}
\label{sec:app}

\subsection*{Simulation Study}


We simulated data for 100 sites over 40 years.  All
simulations assumed initial abundance was Poisson distributed
and no covariates affected initial abundance, dynamics, or
detection probability.  Our first series of simulations
assumed dynamics were exponential.  We ran 1000 simulations for
each combination of low, medium, and high $\Lambda \in
\{1,5,10\}$, $r \in \{-0.005, 0, 0.005\}$, and
$p \in \{0.05, 0.25, 0.5\}$. % (Table 1; 27 total combinations).
Our second series of simulations changed dynamics to the Ricker-logistic
model. We used an initial abundance of 10 and a detection probability
of 0.25, and simulated low, medium, and high values of equilibrium
abundance and maximum growth rate, each with 1000 simulations (Table 1;
9 total combinations). Our third series of simulations was based
on the Ricker-logistic + immigration dynamics model; here we fixed all
parameters the same as the Ricker-logistic model (with $r$ = 0.05 and $K$ = 10) and
simulated low, medium, and high values of immigration rate with 1000
simulation each (Table 1).

We used the \textbf{R} package \texttt{unmarked} to estimate the parameters 
for each simulation with the same
initial abundance (Poisson) and dynamics models as were simulated.
When run in a maximum likelihood framework, these models require a
a finite upper bound $N_{max}$,
\citep{royle:2004biom,dail_madsen:2011};
%(Royle 2004, Dail and Madsen 2011);
we used 200.  We report bias of estimates, root
mean squared error, and coverage (percentage of 95\% confidence
intervals for parameters that overlap the true values).

\begin{comment}
Ideas:
I. Test importance of random effects modeling, and other
differences between JAGS and unmarked
II.	Model with low p values, or a range of p values
III. Ability to distinguish correct initial abundance and
dynamics models
IV.	When can you detect DD if it exists (power)?
V.	Building on the last two: what if your model set doesn't
include the correct DD model? When will the wrong DD model be
favored
over the wrong DID model?
\end{comment}



\subsection*{Analysis of Breeding Bird Survey Data}

We applied our model extensions to North American Breeding Bird Survey
(BBS) data collected from 1966-2010 in %the bordering US states
Maryland and Virginia. For our focal species, we selected the
ovenbird (\textit{Seiurus aurocapilla}), an abundant and widespread,
forest-breeding migratory songbird with a stable or increasing trend
in the region \citep{porneluzi_etal:2011}. 
%and golden-winged warblers (\textit{Vermivora chrysoptera}), which
%only breeds in the western parts of these states and has been declining in the
%region \citep{confer_etal:2011}.  %MORE INFORMATION ABOUT THE SPECIES?

%GENERAL ABOUT BBS DATA.
The BBS is an annual roadside survey implemented by trained
observers in the United States and Canada. An observer conducts 50
3-minute point counts with 400 m radii, 0.8 km apart 
%from each other 
along a
39.4 km route. We only used data marked as acceptable for use in the annual BBS
analysis \citep{sauer_etal:1994auk}.  We summed the number of ovenbirds
seen on each route and year, and used the routes (rather
than the individual stops) as our sites.
Strong winds can interfere with point count observers' ability
to hear birds \citep{simons_etal:2007}; we tested the effects of wind
speed on detection probability, using the truncated mean of wind conditions at the start and end of
each route on the Beaufort scale as a categorical predictor. 
%BBS volunteers record wind conditions
%at the beginning and end of each route on the Beaufort Scale
%\citep[start and end wind]{robbins_etal:1986}.
%%(start and end wind; Robbins et al. 1986)(start and end wind;0-9; cite).
%When start or end wind was not recorded we imputed those values with
%the start or end mean. We took the mean of start and end wind for
%each route and put this average wind scale value into four
%categories: 0 $\leq$ wind $<$ 1; 1 $\geq$ wind $<$ 2; 2 $\leq$ wind $<$ 3; and wind
%$\geq$ 3 (maximum of 3.5).
Following \citet{link_sauer:2002},
%Link and Sauer (2002),
we also included the first time an observer ran a route as a predictor variable for detection
probability.
 
We fitted a series of models using maximum likelihood, 
and then a series of models using MCMC. We started by testing
three models of initial abundance (Poisson, negative binomial, and
zero-inflated Poisson) with exponential dynamics
(Eq.~\ref{eq:exp}) and no covariates.  We selected the minimum Akaike's Information
Criterion (AIC) model from that set to test three additional
models for p: wind, first, and wind + first. We selected the minimum
AIC model from that set to test eight additional models of dynamics:
constant-recruitment, geometric-recruitment, Ricker-logistic, Gompertz-logistic, 
the previous three with immigration, and exponential + immigration.
%geometric-recruitment +
%immigration, exponential + immigration, Ricker-logistic +
%immigration, and Gompertz-logistic + immigration. %We then tested the effect of average
%minimum temperature for June and July on all dynamics parameters from
%the minimum AIC dynamics model.
%As above, t
These models require a a finite upper bound ($N_{max}$) to integrate
over when run in a maximum likelihood
framework; we used 600. % for ovenbirds and 350 for golden-winged warblers.

We ran the top ranked models from the maximum likelihood
analyses in a Bayesian framework with non-informative priors ($\Lambda$,
$\alpha$, and $K  \sim \mathrm{U}(0, 200)$, $r  \sim \mathrm{U}(0, 5)$,
$\iota  \sim \mathrm{U}(0, 15)$, and logit-linear $p$ coefficients $\sim \mathrm{N}(0, 100)$).  
% JEFF: We need to be specific about the priors
We added random observer 
effects and environmental stochasticity, modeled as lognormal
variation in regional population growth rate, Eq.~\ref{eq:nuRand}. % and, where appropriate, zero-inflated dynamics.  
We tested for lack of convergence using
five Markov chains for each model \citep{gelman_rubin:1992}.
For each chain we sampled the MCMC for at least 40,000 iterations, after at
least 3,000 adaptation and tuning samples.  
We also used the random observer effects model to obtain route and 
year specific abundance estimates.  Because of the memory requirements involved,
we used %obtained these estimates from 
a single chain.  %We present these estimates for ovenbirds only.

\section*{Results}

\subsection*{Simulation Study}

Estimator bias and error were generally low with the exponential model, and 
coverage was nominal (93.4\% - 97.0\%; Fig.~\ref{fig:exp_hists}; Table S2.1 in Appendix 2). 
However, estimates of $r$ were 
imprecise (RMSE between 0.020 and 0.025; relative RMSE between 413\% and 
482\% for non-zero true $r$) and biased low (between -0.008 and -0.005; relative bias between
-151\% and -101\% for non-zero true $r$) when the true value of
$\Lambda$ was low. % (1). 
Simulations of the Ricker-logistic model %also 
generally performed well (Fig. S2.1 and Table S2.2 in Appendix 2), except when the true value of $r$ was low (0.005). In this case the 
estimates of $r$ and $K$ were inaccurate (relative RMSE between 87.4\% and 2091.8\%),
biased high (between 65.4\% and 198.9\%), and had low coverage (between 77.6\% and 94.3\%).
In other cases for the Ricker-logistic model, relative RMSE was between 6.7\%
and 55\%, bias was between -6.4\% and 5.8\%, and coverage was between
91.9\% and 96.4\%.  All three cases simulated for the Ricker-logistic + immigration model 
performed fairly well (Fig. S2.2 and Table S2.3 in Appendix 2; relative RMSE between 7.2\% 
and 62.9\%, bias between -3.5\% and 3.3\%, and coverage between
93.2\% and 96.4\%).
  
\subsection*{Analysis of Breeding Bird Survey Data}


The negative binomial distribution was strongly supported for ovenbird
initial abundance over the Poisson and zero-inflated Poisson
(Table S3.1A in Appendix 3). %Even compared to the Poisson, there was
%no evidence to support a zero-inflation factor for this species.
The best supported model for $p$ included additive effects of wind speed
and first run of a route by an observer (Table S3.1B, model B.1). First run 
and increasing wind speeds both decreased $p$. All dynamics models 
with immigration were better supported than models without immigration 
(Table S3.1C); the best supported of these was the Ricker-logistic + immigration 
(see Appendix 3, Fig. S3.1 for estimates).
%This model estimated $\Lambda$ at 33.4 $\pm$ 4.7, $r$ at 0.026 $\pm$ 0.006, 
%$K$ at  56.9 $\pm$ 9.4, and $\iota$ at 0.283 $\pm$ 0.043.
%The best supported model of those without immigration was the
%Gompertz-logistic, and the geometric-recruitment and the constant model had the least
%support.
%There was little support for an effect of minimum June and July
% temperature on dynamics parameters of the Ricker + Immigration model
%(Table 2A, models A.17 - A.18).
%For both species, 
Constant-recruitment dynamics models estimated unrealisticly high
survival probabilities ($\omega$ = 1 $\pm$ 1.3e-05), whereas % for ovenbird and
%0.935 $\pm$ 0.011 for
%golden-winged warbler). 
the geometric-recruitment and geometric-recruitment + immigration
models estimated nonviably low survival probabilities %for ovenbird 
($\omega$ =
0.058 $\pm$ 0.046 and 0.026 $\pm$ 0.060, respectively). 
%but not for golden-winged warbler ($\omega$ = 0.715 $\pm$ 0.288 for both models).

Estimates for the top ranked model %for ovenbird
%(NB[$\Lambda (.) \alpha (.)$]Ricker-logistic+Immigration[$r(.)K(.) \iota(.)$]p(wind+1st)) 
were similar when
run in the Bayesian framework, except that the estimate of $r$
more than halved 
%% JEFF: How do we explain that????
(Appendix 3, Fig. S3.1). %to 0.011 $\pm$ 0.005 [for
%Bayesian model estimates we present mean and SD]). 
When random observer effects
were added, estimates for $\Lambda$ and $K$ increased dramatically
%(to 42.5 $\pm$ 7.3 and 101.8 $\pm$ 21.1,
%respectively),
and the estimate for r was intermediate. The %(0.018 $\pm$ 0.006)
estimate of the intercept for $p$ dropped. % from -1.5 $\pm$
%0.1 to -2.0 $\pm$ 0.1 (on the logit scale), and the estimate of $\sigma_p$ was 0.35 $\pm$ 0.03.  
When environmental stochasticity was added, the estimate for $K$ increased. %(to 112 $\pm$ 58.8) and $r$ was
%again intermediate (0.020 $\pm$ 0.010).  Adding environmental stochasticity alone had little effect
%on estimate of $p$ or $\Lambda$. 
The highest estimates of $\Lambda$ and $K$ came 
from the model that included environmental stochasticity and random observer effects. %(43.8 $\pm$
%7.9 and 141.5 $\pm$ 69.5, respectively).  
Gelman and Rubin diagnostics and visual examination of the chain trajectory and density plots
provided little evidence of lack of convergence for any of the Bayesian models ran.  

We estimated ovenbird abundance by route and year (and averaged over routes) 
using the Bayesian model with random observer effects (Fig.~\ref{fig:oven_N}).  
Estimated average abundance over time alternated between growing and stable periods.  
Estimated ovenbird
abundance was highest in western and eastern Maryland in 1970 and 1980, but became
more evenly distributed by 2010.  
%These aren't up to date, and it doesn't look like people use CV as a measure of 
%uncertainty in MCMC estimates anyway.
%Coefficient of variation (CV) for route and year specific
%abundance estimates varied between 0.11 and 2.31 (mean = 0.46).  By comparison, CV
%for the estimate of mean initial abundance ($\Lambda$) was 0.15.
%As expected, uncertainty in estimates of route and year specific abundance was high.

%There was slightly more support for the ZIP distribution for initial
%abundance of golden-winged warbler than for the negative binomial; both were strongly
%supported over the Poisson (Table 3A). The best
%supported model for $p$ was an effect of first run (Table 3B, model
%B.1). The best supported dynamics model was exponential
%(Table 3C, model C.1). %; the estimate of r from this model was -0.058
%$\pm$ 0.017 (SE). There was no support for immigration added to the
%exponential or geometric-recruitment models (Table 3C, models C.4 and
%C.6).
%However, the Gompertz-logistic + immigration and Ricker-logistic + immigration
%models were supported over the corresponding models without immigration
%(Table 3C, models C.4, C.5, C.7, and C.8), despite low estimates
%of $\iota$ from these models (0.003 $\pm$ 0.003 and 0.003 $\pm$
%0.002, respectively).  
%There was considerable support for an effect of
%average minimum June and July temperature on r in the population
%growth model (Table 1B, model B.17), with a negative slope.
%Because of the closeness of rankings for the ZIP and negative binomial
%distributions for golden-winged warbler, we also ran the subsequent models with the
%negative binomial initial abundance; rankings of subsequent
%models were similar. %, but with a $p$ covariate (first) the negative
%binomial distribution outranked the ZIP.

%Estimates for the top ranked model for golden-winged warbler
%%(ZIP[Λ(.)α(.)]Exponential[r(.)]p(1st))
%were similar when run in the Bayesian framework. When random
%observer effects were added, the estimate for $\Lambda$ increased dramatically,
%and the estimate of the intercept for $p$ dropped.
%%(from 22.4 $\pm$ 7.6 to 55.5 $\pm$ 30.9). The estimate of the intercept
%%for $p$ (on the logit scale) dropped from -3.5 $\pm$ 0.2 to -5.6 $\pm$ 1.1, and the
%%estimate of $\sigma_p$ was 1.7 $\pm$ 0.7. 
%Adding environmental stochasticity also increased the estimate of $\Lambda$ 
%%(to 42.9 $\pm$ 17.9) 
%and the model with both random effects had the highest estimate of $\Lambda$.
%% (74.6 $\pm$ 32.9). 
%Adding zero-inflation to the dynamics (as well as
%the initial abundance) has no effect for exponential models, because populations
%that started at 0 would stay at 0 in any case. Therefore, we tested
%zero-inflated dynamics on the Exponential + Immigration model. Adding zero-inflation to the dynamics
%increased the estimate of $\iota$, but had only minor % from 0 to 0.091 $\pm$ 0.049
%effects on the other parameters. Gelman and Rubin diagnostics and visual
%examination of the chain trajectory and density plots provided evidence for a lack of convergence in
%estimates of $\Lambda$ for all Bayesian models ran, and for most parameters when random observer
%effects were included.

%Estimated detection probabilities were generally low. The estimated probability
%of detecting an ovenbird in the top model (Table 2, model C.1) varied between 0.157
%$\pm$ 0.014 and 0.184 $\pm$ 0.015, depending on wind speed and whether the
%observer had run that route before. The estimated probability of detecting a
%golden-winged warbler went from 0.013 $\pm$ 0.006 for first time observers to 0.028 $\pm$ 0.008 for
%repeat observers for the top model (Table 3, model C.1). Accounting for random
%observer effects decreased average detection probability for both species.





\section*{Discussion}

\begin{comment}
  The model (1) recognizes
the discrete nature of count data and the underlying true
abundance
state, (2) includes a realistic observation model with explicitdetection probability parameters (3) has estimable parameters,including those describing the initial state at time $t=1$, and
(4)
allows for inference about both temporal and spatial variation
in
  abundance.
\end{comment}


Our work %has highlighted the limitations of classical state-space
%models as applied to ecological time series data, and we have
%demonstrated 
demonstrates how a class of open population $N$-mixture models
proposed by \citet{dail_madsen:2011} overcomes 
%these limitations. 
many of the limitations of classical state-space models as applied to
ecological time series data.
We extended their model in several important
ways to accomodate common features of ecological
datasets, including sparse counts, zero-inflation, environmental
stochasticity, and random variation in observation error.
%We made four important developments of the open population
%N-mixture model proposed by \citet{dail_madsen:2011} that make the
%model more applicable to long-term data such as is collected
%by the North American Breeding Bird Survey.
%First, we have
%reconciled the objectives of traditional state-space models
%with open population $N$-mixture models by illustrating how
%classical models of population dynamics can be embedded within
%the framework. Second, we have demonstrated methods of
%accounting for zero-inflation in the time series. Third, we
%have illustrated how additional random effects such as
%observer-specific detection probabilities can be
%accommodated. Fourth, we presented a Bayesian analysis of the
%model, which, makes it possible to incorporate prior information when
%available, and in some cases, facilitates parameter estimation.
We also demonstrated how many of the objectives of conventional
state-space modeling can be accomplished within this expanded
framework. For example, one of the primary aims of our paper was to demonstrate how classical
population growth models can be embedded in the DM model.
Although we view this as an important connection to make between the two
different classes of state-space models, we believe that
%In some ways,
such population growth models are more phenomenological
than mechanistic, and this runs counter to the motivation for
the DM model. However, estimating
demographic parameters from count data is an ambitious goal, and
the required assumptions will not be valid in many cases,
especially when immigration and emigration occur. This was supported by the
unrealistic estimates of vital rates reported here and by the
lack of support for the mechanistic models relative to the population
growth models.

One strategy for adhereing to the original formulation of the
model that included survival and recruitment parameters would be to
use either prior information or additional data. For example, in a
Bayesian analysis existing estimates of demographic parameters
could be used as informative priors, which might make it possible to
seperate the effects of movement versus births and deaths.
Alternatively, ecologists could combine count data with data
from, for example,
capture-recapture data. Models that are fitted to count data and
demographic data are often referred to as integrated population models
\citep[IPM;][]{besbeas_etal:2002,
buckland_etal:2004,schaub_etal:2007}.
We believe that a DM modeling framework could work well in this context,
and that the DM model could be viewed as a specific case of the
IPM proposed by \citet{buckland_etal:2004}.
However, we caution that when count data are affected by
movement and vital rates, it would be unwise to ignore the movement
processes.

%In contrast, recent work has sought to use state-space
%models for inference about population processes such as mortality and
%recruitment. To do so, methods have been developed to combine multiple
%sources of information, such as counts and mark-recapture data
%\citep{besbeas_etal:2002, buckland_etal:2004,
%  schaub_etal:2007}.

Integrated population models may be viewed as the gold standard in
state-space modeling, but ecologists do not always have direct
information about vital rates, especially at large spatial scales.
Count data are much more common, relatively inexpensive to obtain, and
are produced by many of the largest monitoring programs in the
world, such as the BBS. %North American Breeding Bird Survey
%\citep[BBS;][]{robbins_etal:1986}.
In the absence of direct information about demographic parameters, and
when the original assumptions of the model do not hold, our extensions to the
DM model can still be used for important purposes such as predicting future
population size and changes in species distributions.

DM models and our extensions are conceptually simple, but even in
their most basic form they include many random effects (the $N$'s, which
incorporate demographic stochasticity), making it challenging to compute the likelihood 
or implement MCMC algorithms. Nonetheless, in some cases additional random effects
may be of interest.  We have shown a few approaches for incorporating some of these
(environmental stochasticity and random observer effects).  %Other approaches to modeling
%each of these may also be worth trying.  
Although incorporating these random effects can increase
the difficulty of achieving model convergence and decrease estimate precision, they can be important.
For example, the estimate of semi-equilibrium abundance for the ovenbird increased by more than 150\%
when environmental stochasticity and random observer effects were included.  

Another important benefit of this class of models is that it can be
readily implemented using frequentist or Bayesian methods by
ecologists lacking skills in computer programming and advanced
statistics. Appendix 1 includes code to
simulate and fit many of the extensions discussed in this paper.
In spite of the new extensions we have proposed, several aspects of
the model could be improved. First, the precision with which the
parameters of the state process can be estimated ultimately depends
upon how well detection probability is estimated. With
only a single survey per primary period, the information about
detection probability comes from deviations from the parametric
assumptions about population dynamics. Thus, without direct information
about detection probability, the estimates will be determined by the
model's parametric assumptions. Furthermore, properly modeling detection
probability can involve accounting for nuances such as difference
among individuals and probabilities of being available for detection
\citep{nichols_etal:2009}.

Fortunately, incorporating direct information about detection probability 
is straightforward and we recommend that this be done whenever
possible. \citet{dail_madsen:2011} proposed that
replicated counts be conducted within an interval in which the populations can 
be assumed to be closed. A robust design \citep{pollock:1982} could be 
used to combine multiple surveys per
primary period to increase the precision of the estimates. We envision
that multiple other options are available as well, such as removal,
multiple observer, and distance modeling \citep{williams_etal:2002}. These could be
accomplished by extending the open population $N$-mixture model
in exactly the same way as the closed population version has been
extended \citep[e.g.,][]{royle_etal:2004}.

Our emphasis was on increasing the practical utility of this
class of models, and so we avoided several conceptually interesting
extensions that we believe would be computationally prohibitive in many
cases. Nonetheless, we will discuss one---spatially-explict models of
immigration. 
In our models, immigration is currently modeled as independent of
population sizes in other sites. Several alternatives suggest
themselves if we think about movement in a metapopulation
context \citep{hanski:1998}: immigration conditional on total or mean
abundance across sites the previous time step; immigration
conditional on abundance in nearby sites only; and immigration
conditional on abundance at sites as a function of their distance
from the receiving site \citep{hastings:1991,hanski:1998}. These models would
be based on the assumptions that either the study sites
encompass the whole range of the metapopulation or that the
annual changes in the populations sampled are representative of
a larger metapopulation. Initial tests suggest that it would be
possible to fit some or all of these models, at least in the
Bayesian framework.
% Could use a better reference than Hastings 1991 for the total abundance 
% immigration models, but haven't been able to find one.  And haven't found
% any references for nearby site immigration (maybe that's because it's not a great idea).
\begin{comment}



For data sources such as the BBS, where stops are nested within
routes, spatially hierarchical versions of these models could be
the best way to model data from individual stops over many
routes. These models would allow the analyst to account for the
potential spatial dependence of stops within a route, by having
the route as another random effect.

Other models of variability in growth or recruitment (such as
negative binomial or generalized Poisson) could be informative.
Stochasticity (represented as variation in the numbers of
animals occurring in a site and year around what the
deterministic dynamics predict) could be over or underdispersed,
compared to the Poisson distribution. Alternatively,
stochasticity could possibly be applied directly to population
model parameters (such as r and K), instead of numbers of
animals.
\end{comment}

%[paragraph on spatially-explicit dispersal]
%[Other possible extensions we could mention:
%Spatially hierarchical model (so can use individual stops)
%Other models of variability in growth or recruitment (such as negative binomial)
%Add density dependence to survival and recruitment models]
%[Discuss causes of low p values (and other BBS results) here or in BBS
%results?]

Our models generally estimated parameters well in analysis of simulated data.
Two exceptions were the estimates of instantaneous growth rate in exponential model when the 
true value of initial abundance was low, and the estimates of equilibrium abundance and maximum
growth rate in the Ricker-logistic model when the true value of maximum growth rate was low.  
The failure of the first case was likely due to a large number of sites starting at abundance 0 or reaching it early in the simulation.  
%[Richard, maybe come up with a summary statistic for this based on the simulation N values?]
Because a site that hits abundance 0 in this simulation stayed at 0, this provided little information with
which to estimate growth rate.  The failure of the second case is also reasonable: with very
slow movement toward the equilibrium abundance comes little information for estimating
that abundance or the rate of growth.  In other cases, bias was low and coverage nominal.

%We have presented a test case for these models using data from
%the BBS for ovenbirds. We demonstrate that these models can be
%used to model and estimate initial abundance, detection
%probability, and density-independent and dependent functions for
%population dynamics. We tested these models in both the
%frequentist and Bayesian frameworks and found strengths and
%weaknesses of each approach. The Bayesian analyses permit more
%flexibility and allow the incorporation of random observer
%effects, environmental stochasticity, and zero-inflated dynamics. 
%Our results suggest ignoring
%variation between observers can lead to underestimation of
%initial abundance and equilibrium population sizes. On the other
%hand, our Bayesian analyses did have some convergence problems.
%%particularly for the species (golden-winged warbler) with sparser data. 
%It's likely that poor model fits are related to the low estimates of
%detection probability, which in turn are probably caused by the
%large 400 m radius used for BBS point counts. Again, we believe
%that direct information on detectability (such as repeated
%visits within a season) would greatly improve the fit of these
%models in both frameworks.  We also were able to obtain year
%and site-specific estimates of abundance (Fig.~\ref{fig:oven_N}),
%although with generally lower precision than more global estimates.  These sort of estimates
%may be more reliable when abundance and dynamics are modeled 
%as functions of habitat covariates, which we did not do here.
\begin{comment}
[Alternatively, could put discussions of our results where
they fit in by subject area, rather than separately. For
instance, could discuss BBS population dynamics model selection
in population models paragraph, BBS Bayesian vs.
frequentist strengths and weakness in the Bayesian paragraph, 
and golden-winged warbler ZIP results in ZIP
paragraph.]
\end{comment}

The modeling framework we described can be used and extended to address many
pressing issues in ecology and conservation biology. For
example,
%it is possible to test hypotheses about temporal and spatial
%population regulation. Furthermore, the Bayesian approach is very
%useful in that it can be used to combine multiple sources of data to
%develop mechanistic models of population dynamics. Thus one can test
it allows for inference %hypotheses
about the effects of climate change on either explicit
demographic parameters or in derived parameters such as
population growth rate. Furthermore, %under the Bayesian approach, population
in addition to assessing past climate change on population
parameters, projecting populations under future climate scenarios is also
possible. Thus, we expect this modeling methodology can be used
to make quantiative assessments of species vulnerability to climate
change and other important human-induced drivers of population
dynamics.


\section*{Acknowledgements}

We thank J. Andrew Royle and T. Scott Sillett for helpful suggestions on our analysis
and manuscript. We gratefully acknowledge the U. S. National Park 
Service, the Smithsonian Institution, and USGS's Breeding Bird Survey program for funding support.
We thank K. Pardieck and D. Zilowkowski for help with data acquisition. J. Sauer 
provided helpful feedback about our analysis.

\bibliography{Dail_Mad_ext}


\newpage

% Tables


\begin{table}[t]
  \centering
\caption{Changed parameter values, by series of simulations.  We
simulated 1000 sets of data each combination of parameter values
for 100 sites over 40 years.  We assumed that initial abundance
($\Lambda$) was Poisson distributed.  For the exponential 
simulations we included all combinations of low, medium, and
high $\Lambda$, growth rate ($r$), and detection probability ($p$).
For the Ricker-logistic model simulations we used $\Lambda$ = 10 and $p$ = 0.25, and
simulated low, medium, and high values of $r$ and equilibrium density ($K$).
For the Ricker-logistic + immigration dynamics model we fixed all parameters the
same as the Ricker-logistic model (with $r$ = 0.05 and $K$ = 10) and
simulated low, medium, and high values of immigration rate ($\iota$).}  
\begin{tabular}{lcccccccc}
    \hline
    & \multicolumn{3}{c}{Exponential Growth} && \multicolumn{2}{c}{Ricker-logistic} &&
    Ricker-logistic + Immigration \\
    \cline{2-4}     \cline{6-7}    \cline{9-9}
% & \multicolumn{3}{c}{\rule{25mm}{.1pt}} & \rule{5cm}{.5pt} & \multicolumn{2}{c}{\rule{20mm}{.1pt}} \\
& $\Lambda$ & $r$ & $p$ && $r$  & $K$ && $\iota$  \\    
\hline
    Low	        &1	&-0.01	&0.05	&&0.005	 &5	&&0.005  \\
    Med	        &5	&0	&0.25	&&0.05	&10	&&0.05   \\
    High		 &10 &0.005	&0.5	&&0.1	&20	&&0.5    \\
    \hline
  \end{tabular}
\end{table}

%\vfill
%\clearpage
\newpage

% Perhaps stack the species
\begin{comment}
\begin{sidewaystable}
  \centering
  \small
\caption{Model selection table for ovenbirds and golden-winged
warblers in Maryland and Virginia, 1966-2010. We present model
name
and number, number of parameters (Par.), and difference in
Akaike's
information criterion between each model and the top model of
that
set ($\Delta$AIC).  For each species, the first section comparesmodels for initial abundance, the second for detection
probability,
    and the third for dynamics.}
  \begin{tabular}[h]{lcclcc}
\hline
A. Ovenbirds	&&&		B. Golden-Winged Warblers \\
\hline
Model	&Par.	&$\Delta$AIC	&Model	&Par.	&$\Delta$AIC \\
A.1. NB[$\Lambda$(.)$\alpha$(.)]Exponential[$r$(.)]$p$(.)	&4	&0
&B.1. ZIP[$\Lambda$(.)$\psi$(.)]Growth[$r$(.)]$p$(.)	&4	&0 \\
A.2. P[$\Lambda$(.)]Growth[$r$(.)]$p$(.)	&3	&1262.7
&B.2. NB[$\Lambda$(.)$\alpha$(.)]Growth[$r$(.)]$p$(.)	&4	&0.4 \\A.3. ZIP[$\Lambda$(.)$\psi$(.)]Growth[$r$(.)]$p$(.)	&4
&1264.7	&B.3. P[$\Lambda$(.)]Growth[$r$(.)]$p$(.)	&3	&37.8 \\
A.4. NB[$\Lambda$(.)$\alpha$(.)]Growth[$r$(.)]$p$(wind+1st)	&8
&0	&B.4. ZIP[$\Lambda$(.)$\psi$(.)]Growth[$r$(.)]$p$(1st)	&5
&0 \\
A.5. NB[$\Lambda$(.)$\alpha$(.)]Growth[$r$(.)]$p$(wind)	&7	&0.9
&B.5. ZIP[$\Lambda$(.)$\psi$(.)]Growth[$r$(.)]$p$(.)	&4	&3.5 \\
A.6. NB[$\Lambda$(.)$\alpha$(.)]Growth[$r$(.)]$p$(1st)	&5	&5.0
&B.6. ZIP[$\Lambda$(.)$\psi$(.)]Growth[$r$(.)]$p$(wind+1st)	&8
&5.9 \\
A.7. NB[$\Lambda$(.)$\alpha$(.)]Growth[$r$(.)]$p$(.)	&4	&6.4
&B.7. ZIP[$\Lambda$(.)$\psi$(.)]Growth[$r$(.)]$p$(wind) &7 &9.3
\\
A.8.
NB[$\Lambda$(.)$\alpha$(.)]Ricker-logistic+Immigration[$r$(.)$K$(.)$\iota$(.)]$p$(wind+1st)
&10	&0	&B.8. ZIP[$\Lambda$(.)$\psi$(.)]Growth[$r$(.)]$p$(1st)
&5	&0 \\
A.9.
NB[$\Lambda$(.)$\alpha$(.)]Gompertz-logistic+Immigration[$r$(.)$K$(.)$\iota$(.)]$p$(wind+1st)
&10	&8.4
&B.9.
ZIP[$\Lambda$(.)$\psi$(.)]Geometric-recruitment[$\gamma$(.)$\omega$(.)]$p$(1st)
&6	&1.7 \\
A.10.
NB[$\Lambda$(.)$\alpha$(.)]Growth+Immigration[$r$(.)$\iota$(.)]$p$(wind+1st)
&9	&36.5
&B.10.
ZIP[$\Lambda$(.)$\psi$(.)]Growth+Immigration[$r$(.)$\iota$(.)]$p$(1st)
&6	&2.0 \\
A.11.
NB[$\Lambda$(.)$\alpha$(.)]Geometric-recruitment+Immigration[$\gamma$(.)$\omega$(.)$\iota$(.)]$p$(wind+1st)
&10	&38.6
&B.11.
ZIP[$\Lambda$(.)$\psi$(.)]Gompertz-logistic+Immigration[$r$(.)$K$(.)$\iota$(.)]$p$(1st)
&7	&2.5 \\
A.12.
NB[$\Lambda$(.)$\alpha$(.)]Gompertz-logistic[$r$(.)$K$(.)]$p$(wind+1st)
&9	&192.8
&B.12. ZIP[$\Lambda$(.)$\psi$(.)]Gompertz-logistic[$r$(.)$K$(.)]$p$(1st)
&6
&3.0 \\
A.13.
NB[$\Lambda$(.)$\alpha$(.)]Ricker-logistic[$r$(.)$K$(.)]$p$(wind+1st)
&9	&195.1
&B.13.
ZIP[$\Lambda$(.)$\psi$(.)]Geometric-recruitment+Immigration[$\gamma$(.)$\omega$(.)$\iota$(.)]$p$(1st)
&7	&3.7 \\
A.14. NB[$\Lambda$(.)$\alpha$(.)]Growth[$r$(.)]$p$(wind+1st)	&8
&271.3
&B.14.
ZIP[$\Lambda$(.)$\psi$(.)]Ricker-logistic+Immigration[$r$(.)$K$(.)$\iota$(.)]$p$(1st)
&7	&4.2 \\
A.15.
NB[$\Lambda$(.)$\alpha$(.)]Geometric-recruitment[$\gamma$(.)$\omega$(.)]$p$(wind+1st)
&9	&273.7
&B.15. ZIP[$\Lambda$(.)$\psi$(.)]Ricker-logistic[$r$(.)$K$(.)]$p$(1st)	&7&5.2 \\
A.16.
NB[$\Lambda$(.)$\alpha$(.)]Constant-recruitment[$\gamma$(.)$\omega$(.)]$p$(wind+1st)
&9	&1856.7
&B.16.
ZIP[$\Lambda$(.)$\psi$(.)]Constant-recruitment[$\gamma$(.)$\omega$(.)]$p$(1st)
&6	&12.5 \\
\hline
\end{tabular}
\end{sidewaystable}
\end{comment}

% How about separate tables?  Also updated the Gompertz-logistic results.
%\begin{table}
%  \centering
%  \small
%\caption{Model selection table for ovenbirds in Maryland and Virginia,
%    1966-2010.  We present model name and number, number of 
%parameters (Par.), and difference in Akaike's
%information criterion between each model and the top model of
%that set ($\Delta$AIC).  The first section compares
%models for initial abundance, the second for detection
%probability, and the third for dynamics.}
%  \begin{tabular}[h]{lcc}
%\hline
%Model	&Par.	&$\Delta$AIC	\\
%\hline
%A. Initial Abundance && \\
%A.1. NB[$\Lambda$(.)$\alpha$(.)]Exponential[$r$(.)]$p$(.)	&4	&0\\
%A.2. P[$\Lambda$(.)]Exponential[$r$(.)]$p$(.)	&3	&1262.7\\
%A.3. ZIP[$\Lambda$(.)$\psi$(.)]Exponential[$r$(.)]$p$(.)	&4 &1264.7\\
%\hline
%B. Detection Probability && \\
%B.1. NB[$\Lambda$(.)$\alpha$(.)]Exponential[$r$(.)]$p$(wind+1st)	&8 &0	\\
%B.2. NB[$\Lambda$(.)$\alpha$(.)]Exponential[$r$(.)]$p$(wind) &7 &0.9\\
%B.3. NB[$\Lambda$(.)$\alpha$(.)]Exponential[$r$(.)]$p$(1st)	&5	&5.0
%\\B.4. NB[$\Lambda$(.)$\alpha$(.)]Exponential[$r$(.)]$p$(.)	&4	&6.4\\
%\hline
%C. Dynamics && \\
%C.1. NB[$\Lambda$(.)$\alpha$(.)]Ricker-logistic+Immigration[$r$(.)$K$(.)$\iota$(.)]$p$(wind+1st) &10	&0	\\
%C.2. NB[$\Lambda$(.)$\alpha$(.)]Gompertz-logistic+Immigration[$r$(.)$K$(.)$\iota$(.)]$p$(wind+1st) &10	&8.4 \\
%C.3. NB[$\Lambda$(.)$\alpha$(.)]Exponential+Immigration[$r$(.)$\iota$(.)]$p$(wind+1st) &9	&36.5\\
%C.4. NB[$\Lambda$(.)$\alpha$(.)]Geometric-recruitment+Immigration[$\gamma$(.)$\omega$(.)$\iota$(.)]$p$(wind+1st) &10	&38.6\\
%C.5. NB[$\Lambda$(.)$\alpha$(.)]Gompertz-logistic[$r$(.)$K$(.)]$p$(wind+1st) &9	&192.8\\
%C.6. NB[$\Lambda$(.)$\alpha$(.)]Ricker-logistic[$r$(.)$K$(.)]$p$(wind+1st) &9	&195.1\\
%C.7. NB[$\Lambda$(.)$\alpha$(.)]Exponential[$r$(.)]$p$(wind+1st)	&8 &271.3\\
%C.8. NB[$\Lambda$(.)$\alpha$(.)]Geometric-recruitment[$\gamma$(.)$\omega$(.)]$p$(wind+1st) &9	&273.7\\
%C.9. NB[$\Lambda$(.)$\alpha$(.)]Constant-recruitment[$\gamma$(.)$\omega$(.)]$p$(wind+1st) &9	&1856.7\\
%\hline
%\end{tabular}
%\end{table}

%\vfill
%%\clearpage
%\newpage

%\begin{table}
%  \centering
%  \small
%  \caption{Model selection table for golden-winged
%warblers in Maryland and Virginia, 1966-2010. We present model
%name and number, number of parameters (Par.), and difference in
%Akaike's information criterion between each model and the top model of
%that set ($\Delta$AIC).  The first section compares
%models for initial abundance, the second for detection
%probability, and the third for dynamics.}
%  \begin{tabular}[h]{lcc}
%\hline
%Model	&Par.	&$\Delta$AIC\\
%\hline
%A. Initial Abundance && \\
%A.1. ZIP[$\Lambda$(.)$\psi$(.)]Exponential[$r$(.)]$p$(.)	&4	&0 \\
%A.2. NB[$\Lambda$(.)$\alpha$(.)]Exponential[$r$(.)]$p$(.)	&4	&0.4 \\
%A.3. P[$\Lambda$(.)]Exponential[$r$(.)]$p$(.)	&3	&37.8 \\
%\hline
%B. Detection Probability && \\
%B.1. ZIP[$\Lambda$(.)$\psi$(.)]Exponential[$r$(.)]$p$(1st)	&5 &0 \\
%B.2. ZIP[$\Lambda$(.)$\psi$(.)]Exponential[$r$(.)]$p$(.)	&4	&3.5 \\
%B.3. ZIP[$\Lambda$(.)$\psi$(.)]Exponential[$r$(.)]$p$(wind+1st)	&8 &5.9 \\
%B.4. ZIP[$\Lambda$(.)$\psi$(.)]Exponential[$r$(.)]$p$(wind) &7 &9.3\\
%\hline
%C. Dynamics && \\
%C.1. ZIP[$\Lambda$(.)$\psi$(.)]Exponential[$r$(.)]$p$(1st) &5	&0 \\
%C.2. ZIP[$\Lambda$(.)$\psi$(.)]Geometric-recruitment[$\gamma$(.)$\omega$(.)]$p$(1st) &6	&1.7 \\
%C.3. ZIP[$\Lambda$(.)$\psi$(.)]Exponential+Immigration[$r$(.)$\iota$(.)]$p$(1st) &6	&2.0 \\
%C.4. ZIP[$\Lambda$(.)$\psi$(.)]Gompertz-logistic+Immigration[$r$(.)$K$(.)$\iota$(.)]$p$(1st) &7	&2.5 \\
%C.5. ZIP[$\Lambda$(.)$\psi$(.)]Gompertz-logistic[$r$(.)$K$(.)]$p$(1st)	&6&3.0 \\
%C.6. ZIP[$\Lambda$(.)$\psi$(.)]Geometric-recruitment+Immigration[$\gamma$(.)$\omega$(.)$\iota$(.)]$p$(1st) &7	&3.7 \\
%C.7. ZIP[$\Lambda$(.)$\psi$(.)]Ricker-logistic+Immigration[$r$(.)$K$(.)$\iota$(.)]$p$(1st) &7	&4.2 \\
%C.8. ZIP[$\Lambda$(.)$\psi$(.)]Ricker-logistic[$r$(.)$K$(.)]$p$(1st)	&7 &5.2 \\
%C.9. ZIP[$\Lambda$(.)$\psi$(.)]Constant-recruitment[$\gamma$(.)$\omega$(.)]$p$(1st) &6	&12.5 \\
%\hline
%\end{tabular}
%\end{table}










%%% Figures
\clearpage

\section*{Figure Legends}
\noindent Figure ~\ref{fig:exp_hists}. Histograms of 1000 parameter estimates for each of 27
simulation cases with exponential dynamics. The vertical lines are the 
data-generating values.

%\noindent Figure ~\ref{fig:rick_hists}. 
%
%\noindent Figure ~\ref{fig:ricki_hists}. Histograms of 1000 parameter estimates for each of 3
%simulation cases with Ricker-logistic + immigration dynamics. The vertical lines are the 
%data-generating values.

\noindent Figure ~\ref{fig:oven_N}. Maps of BBS route and year specific estimated 
abundances for ovenbirds, for the years 1970, 1980, 1990, 2000, and 2010.  
Lower right panel depicts estimated mean route abundance by year
(mean and 95\% credible interval).

\begin{figure}
\caption{}
  \centering
  \includegraphics[height=8in]{figs/exp_hists}
\label{fig:exp_hists}
\end{figure}

\begin{figure}
\caption{}
  \centering
  \includegraphics[width=6.6in]{figs/OVEN_N_by_route_year6}
\label{fig:oven_N}
\end{figure}
\end{flushleft}
\end{spacing}
\end{document}
